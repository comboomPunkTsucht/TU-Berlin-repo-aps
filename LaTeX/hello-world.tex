% !TEX program = lualatex
\documentclass[12pt,a4paper]{report} % oder book

\usepackage{fontspec}    % bei lualatex/xelatex
\usepackage{microtype}
\usepackage{graphicx}
\usepackage{amsmath,amssymb}
\usepackage[ngerman]{babel}       % deutsche Silbentrennung
\usepackage[hidelinks]{hyperref}
\usepackage{bookmark}    % besseres TOC → hyperref
\usepackage{fancyhdr}

% Versuche zuerst die gewünschte Systemschrift, sonst Fallback
\IfFontExistsTF{CaskaydiaCove Nerd Font Propo}
  {\setmainfont{CaskaydiaCove Nerd Font Propo}}
  {\setmainfont{Latin Modern Roman}} % Fallback

% Fancyhdr: Seitenzahlen rechts in Fußzeile (du kannst C->zentriert ändern)
\pagestyle{fancy}
\fancyhf{}                        
\fancyfoot[C]{\thepage}
\renewcommand{\headrulewidth}{0pt} 
\renewcommand{\footrulewidth}{0pt}

\newcommand{\titleline}{Titel der Arbeit}
\newcommand{\subtitleline}{Subtitle der Arbeit}
\newcommand{\authorsline}{Fabian Aps}
\newcommand{\matrikelline}{525528}


% Metadaten für PDF
\hypersetup{
  pdftitle={\titleline},
  pdfauthor={\authorsline - \matrikelline},
  pdfsubject={\subtitleline},
  pdfcreator={LuaLaTeX}
}

\begin{document}

% ------------------
% Titelseite (keine Seitenzahl)
% ------------------
\begin{titlepage}
  \centering
  {\Large Technische Universität Berlin\par}
  \vspace{2cm}
  {\Huge\bfseries \titleline\par}
  \vspace{1.5cm}
  {\Large \subtitleline\par}
  \vfill
  {\large Autor: \authorsline\par}
  {\large Matrikelnummer:\matrikelline\par}
  {\large Studium: B.Sc. Informatik\par}
  \vspace{1cm}
  {\large Betreuer: Prof. Dr. X\par}
  \vspace{2cm}
  {\large \today\par}
  \thispagestyle{empty} % Titelseite ohne Seitenzahl
\end{titlepage}

% ------------------
% Inhaltsverzeichnis mit römischen Seitenzahlen (i, ii, ...)
% ------------------
\cleardoublepage
\pagenumbering{roman} % i, ii, iii, iv ...
\setcounter{page}{1}  % falls gewünscht

\tableofcontents
\clearpage

% ------------------
% Hauptteil: arabische Seitenzahlen ab 1
% ------------------
\cleardoublepage
\pagenumbering{arabic}
\setcounter{page}{1}

\chapter{Einleitung}\label{ch:einleitung}
Hier beginnt der Hauptteil.
Seitenzahlen erscheinen nun als 1, 2, 3, \ldots

\section{Beispiel}\label{sec:beispiel}
Text\ldots

\end{document}

% Local Variables:
% TeX-engine: luatex
% End:
