\documentclass[12pt]{article}
\usepackage[utf8]{inputenc}
\usepackage{amsmath}
\usepackage{amssymb}
\usepackage{graphicx}
\usepackage{lmodern}
\usepackage{bm}
\usepackage[ngerman]{babel}
\allowdisplaybreaks
\setlength{\parindent}{0.0em}
\usepackage[left=3.5cm,right=3.5cm,top=3cm,bottom=3cm]{geometry}
\usepackage{hyperref} %Package, um den Link unten in diese Vorlage einzufügen, kann also in Hausaufgaben weggelassen werden

\newcommand*{\QED}{\hfill\ensuremath{\square}} 

\newenvironment{aufgabe}[1]{\noindent\textbf{#1.\,Aufgabe}\\[.5ex]}{\ \\[1ex]}
\newenvironment{loesung}[1]{\noindent\textbf{Lösung zu #1.\,Aufgabe}\\[.5ex]}{\vfill}

\title{Analysis I und Lineare Algebra für Ingenieurwissenschaften WS 21/22\\ 01.\,Hausaufgabenblatt}
\author{Gruppe: Zama A\\ Erlenbauer, Nada, Matrikelnummer \\  Al Mashlab, Tarek, Matrikelnummer\\ Vorderwülbecke, Luca, Matrikel-Nr.465798}

\date{\today}

\begin{document}
\maketitle


\section*{Hausaufgaben:}

\begin{aufgabe}{2}
	Es sei $x\in\mathbb{R}$. Zeigen Sie, dass für alle $n\in\mathbb{N}$ gilt, dass $x^n\in\mathbb{R}$.
\end{aufgabe}

\begin{loesung}{2}
	\textbf{Induktionsanfang:}
	Nach Voraussetzung ist $x^1=x\in\mathbb{R}$. Damit gilt schon einmal der Induktionsanfang.\\[0.5ex]
	\textit{Induktionsvoraussetzung:}
	Sei nun $x^n\in\mathbb{R}$ für ein beliebiges aber festes $n\in\mathbb{N}$.\\[.5ex]
	\underline{Induktionsschritt $(n\to n+1)$}: Ist die Induktionsvoraussetzung für $n\in\mathbb{N}$ erfüllt, so gilt, dass
	\[
	x^{n+1}
	= x^n\cdot x
	\in \mathbb{R}
	\]
	sein muss, da sowohl $x^n\in\mathbb{R}$ als auch $x\in\mathbb{R}$ und $\mathbb{R}$ abgeschlossen ist unter Multiplikation. \hfill $\square$
\end{loesung}

\begin{aufgabe}{3}
	Es sei $f:\mathbb{R} \to \mathbb{R}, x\mapsto x^2\cdot e^{-x}$. Bestimmen Sie die erste Ableitung von $f$.
\end{aufgabe}

\begin{loesung}{3}
	Beide Abbildungen $x\mapsto x^2$ und $x\mapsto e^{-x}$ sind differenzierbar. Daher folgt nach der Produktregel:
	\begin{align*}
		f'(x)
		&= (x^2)'\cdot e^{-x} + x^2\cdot(e^{-x})'\\[2.5ex]
		&= 2xe^{-x} + x^2(-e^{-x})
		= (2x-x^2)e^{-x}.
	\end{align*}
	\hfill $\square$
\end{loesung}


\end{document}