% !TEX program = lualatex
\documentclass[12pt,a4paper]{report}

\usepackage{fontspec}
\usepackage{microtype}
\usepackage{graphicx}
\usepackage{amsmath,amssymb}
\usepackage[ngerman]{babel}
\usepackage[hidelinks]{hyperref}
\usepackage{bookmark}
\usepackage{fancyhdr}
\usepackage{nicefrac}
\usepackage{caption}
\usepackage{subcaption}

% Seitenränder anpassen
\usepackage[a4paper, margin=2.5cm]{geometry}

% Schriftart-Einstellungen
\IfFontExistsTF{CaskaydiaCove Nerd Font Propo}
  {\setmainfont{CaskaydiaCove Nerd Font Propo}}
  {\setmainfont{Latin Modern Roman}}

% Kopf- und Fußzeilen
\pagestyle{fancy}
\fancyhf{}
\fancyfoot[C]{\thepage}
\renewcommand{\headrulewidth}{0pt}
\renewcommand{\footrulewidth}{0pt}

% Metadaten
\newcommand{\titleline}{Ana1LinA - Hausaufgabe 08}
\newcommand{\subtitleline}{Felix 09}
\newcommand{\authorsline}{Fabian Aps:525528, Joris Victor Vorderwülbecke:0528715, Emil Arthur Joseph Hartmann:052542, Friedrich Ludwig Finck:0526329}

\hypersetup{
  pdftitle={\titleline},
  pdfauthor={\authorsline},
  pdfsubject={\titleline - \subtitleline},
  pdfcreator={LuaLaTeX}
}

\begin{document}

% ------------------
% Titelseite
% ------------------
\begin{titlepage}
  \centering
  {\Large Technische Universität Berlin\par}
  \vspace{2cm}
  {\Huge\bfseries \titleline\par}
  \vspace{1.5cm}
  {\Large \subtitleline\par}
  \vfill
  {\large Autoren: \authorsline\par}
  \vspace{2cm}
  {\large \today\par}
  \thispagestyle{empty}
\end{titlepage}

% ------------------
% Inhaltsverzeichnis
% ------------------
\cleardoublepage
\pagenumbering{roman}
\setcounter{page}{1}

\tableofcontents
\clearpage

% ------------------
% Hauptteil
% ------------------
\cleardoublepage
\pagenumbering{arabic}
\setcounter{page}{1}

\chapter{Aufgabe} \label{ch:aufgabe}

\textbf{Geben Sie diese Hausaufgabe in schriftlicher Form in ISIS ab.}

\begin{enumerate}
    \item[a)] Zeigen Sie mit dem Zwischenwertsatz, dass die Gleichung
    $$
    \frac{(x-1)e^x}{x^2-3} = \sin(x)
    $$
    mindestens eine reelle Lösung im Intervall $[-1, \frac{3}{2}]$ besitzt.

    \item[b)] Bestimmen Sie die Ableitungen der Funktion mit der Definition (also als Grenzwert des Differenzenquotienten).
    $$
    f : \mathbb{R} \to \mathbb{R}, \quad f(x) = (x-1)|x-1|
    $$

    \item[c)] Existiert der folgende Grenzwert? Bestimmen Sie gegebenenfalls den Grenzwert mit der Regel von Bernoulli/de l'Hospital.
    $$
    \lim_{x \searrow 0} \frac{\ln(\cos(x))}{\ln(\cos(3x))}
    $$
\end{enumerate}

\vspace{1cm}

\begin{figure}[h!]
    \centering
    % Stellen Sie sicher, dass 'ha_08.png' im selben Verzeichnis liegt wie die .tex Datei
    \includegraphics[width=0.95\textwidth]{ha_08.png}
    \caption*{Original Aufgabenstellung}
\end{figure}

% ------------------
% Lösungen
% ------------------

\chapter{Lösung zur Aufgabe} \label{ch:loesung}

\section*{Zu Teilaufgabe a)}

Wir betrachten die Hilfsfunktion $h: D \to \mathbb{R}$, definiert durch:
$$
h(x) = \frac{(x-1)e^x}{x^2-3} - \sin(x)
$$
Gesucht ist eine Nullstelle von $h(x)$ im Intervall $I = [-1, \frac{3}{2}]$.

\subsection*{1. Definitionsbereich und Stetigkeit}
Der Nenner $x^2-3$ wird null bei $x = \pm\sqrt{3}$. Da $\sqrt{3} \approx 1,73 > 1,5$, liegt keine Polstelle im betrachteten Intervall $[-1, 1,5]$.
Die Funktion setzt sich aus stetigen elementaren Funktionen (Polynome, Exponentialfunktion, Sinus) zusammen und ist somit auf dem Intervall $I$ stetig (vgl. Satz 19.10 im Skript). Dies ist die Voraussetzung für die Anwendung des Zwischenwertsatzes.

\subsection*{2. Überprüfung der Funktionswerte an den Rändern}
Wir berechnen $h(-1)$ und $h(1,5)$.

\textbf{Für $x = -1$:}
\begin{align*}
h(-1) &= \frac{(-1-1)e^{-1}}{(-1)^2-3} - \sin(-1) \\
      &= \frac{-2e^{-1}}{1-3} + \sin(1) \\
      &= \frac{-2e^{-1}}{-2} + \sin(1) \\
      &= \frac{1}{e} + \sin(1)
\end{align*}
Da $e > 0$ und $\sin(1) > 0$ (da $1$ im Bogenmaß in $[0, \pi]$ liegt), folgt $h(-1) > 0$.

\textbf{Für $x = \frac{3}{2} = 1,5$:}
\begin{align*}
h(1,5) &= \frac{(1,5-1)e^{1,5}}{1,5^2-3} - \sin(1,5) \\
       &= \frac{0,5 \cdot e^{1,5}}{2,25-3} - \sin(1,5) \\
       &= \frac{0,5 \cdot e^{1,5}}{-0,75} - \sin(1,5)
\end{align*}
Der erste Term ist negativ (Zähler positiv, Nenner negativ). Da $\sin(1,5) \approx 1 > 0$, wird eine positive Zahl subtrahiert. Somit ist $h(1,5)$ die Summe zweier negativer Zahlen, also gilt $h(1,5) < 0$.

\subsection*{3. Fazit}
Da $h$ stetig ist und $h(1,5) < 0 < h(-1)$ gilt, existiert nach dem Zwischenwertsatz (vgl. Satz 20.4 im Skript) mindestens ein $\xi \in [-1, \frac{3}{2}]$ mit $h(\xi) = 0$. Dies entspricht einer Lösung der ursprünglichen Gleichung.

\section*{Zu Teilaufgabe b)}

Die Funktion ist definiert als $f(x) = (x-1)|x-1|$. Wir untersuchen die Ableitung $f'(x_0)$ mittels des Differentialquotienten (vgl. Definition 21.1 im Skript):
$$
f'(x_0) = \lim_{k \to 0} \frac{f(x_0+k) - f(x_0)}{k}
$$

Wir unterscheiden drei Fälle für $x_0$:

\textbf{Fall 1: $x_0 > 1$} \\
Hier ist $x-1 > 0$, also $|x-1| = x-1$ und somit $f(x) = (x-1)^2$.
\begin{align*}
\lim_{k \to 0} \frac{(x_0+k-1)^2 - (x_0-1)^2}{k}
&= \lim_{k \to 0} \frac{(x_0-1)^2 + 2k(x_0-1) + k^2 - (x_0-1)^2}{k} \\
&= \lim_{k \to 0} \frac{2k(x_0-1) + k^2}{k} \\
&= \lim_{k \to 0} (2(x_0-1) + k) \\
&= 2(x_0-1)
\end{align*}

\textbf{Fall 2: $x_0 < 1$} \\
Hier ist $x-1 < 0$, also $|x-1| = -(x-1)$ und somit $f(x) = -(x-1)^2$.
\begin{align*}
\lim_{k \to 0} \frac{-(x_0+k-1)^2 - (-(x_0-1)^2)}{k}
&= \lim_{k \to 0} \frac{-[ (x_0-1)^2 + 2k(x_0-1) + k^2 ] + (x_0-1)^2}{k} \\
&= \lim_{k \to 0} \frac{-2k(x_0-1) - k^2}{k} \\
&= -2(x_0-1)
\end{align*}

\textbf{Fall 3: $x_0 = 1$} \\
Hier ist $f(1) = 0$. Wir prüfen den Differenzenquotienten:
\begin{align*}
\lim_{k \to 0} \frac{f(1+k) - f(1)}{k}
&= \lim_{k \to 0} \frac{(1+k-1)|1+k-1| - 0}{k} \\
&= \lim_{k \to 0} \frac{k|k|}{k} \\
&= \lim_{k \to 0} |k| = 0
\end{align*}
Somit ist $f'(1) = 0$.

\textbf{Ergebnis:}
Zusammenfassend lässt sich die Ableitung schreiben als:
$$
f'(x) = 2|x-1|
$$

\section*{Zu Teilaufgabe c)}

Wir untersuchen den Grenzwert:
$$
L = \lim_{x \searrow 0} \frac{\ln(\cos(x))}{\ln(\cos(3x))}
$$
Für $x \to 0$ gilt $\cos(x) \to 1$ und $\cos(3x) \to 1$. Da $\ln(1) = 0$, liegt ein unbestimmter Ausdruck vom Typ ,,$\frac{0}{0}$`` vor. Die Voraussetzungen für die Regel von Bernoulli/de l'Hospital sind erfüllt (vgl. Satz 22.7 im Skript).

Wir leiten Zähler und Nenner ab (Kettenregel):
\begin{itemize}
    \item Zähler: $(\ln(\cos(x)))' = \frac{1}{\cos(x)} \cdot (-\sin(x)) = -\tan(x)$
    \item Nenner: $(\ln(\cos(3x)))' = \frac{1}{\cos(3x)} \cdot (-\sin(3x) \cdot 3) = -3\tan(3x)$
\end{itemize}

Anwendung von l'Hospital:
$$
L = \lim_{x \searrow 0} \frac{-\tan(x)}{-3\tan(3x)} = \frac{1}{3} \lim_{x \searrow 0} \frac{\tan(x)}{\tan(3x)}
$$
Dies ist erneut ein Ausdruck vom Typ ,,$\frac{0}{0}$``. Wir verwenden den bekannten Grenzwert $\lim_{x \to 0} \frac{\tan(x)}{x} = 1$ (oder wenden l'Hospital erneut an):
\begin{align*}
L &= \frac{1}{3} \lim_{x \searrow 0} \left( \frac{\tan(x)}{x} \cdot \frac{3x}{\tan(3x)} \cdot \frac{1}{3} \right) \\
  &= \frac{1}{3} \cdot 1 \cdot 1 \cdot \frac{1}{3} \\
  &= \frac{1}{9}
\end{align*}

Der Grenzwert existiert und beträgt $\frac{1}{9}$.

\end{document}

% Local Variables:
% TeX-engine: luatex
% End:
