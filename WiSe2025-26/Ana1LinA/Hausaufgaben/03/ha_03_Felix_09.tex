% !TEX program = lualatex
\documentclass[12pt,a4paper]{article} % oder book

\usepackage{fontspec}        % bei lualatex/xelatex
\usepackage{microtype}
\usepackage{graphicx}
\usepackage{amsmath,amssymb}
%\usepackage{enumitem}        % Für [(a)]-Listen und \label in Listen
\usepackage[ngerman]{babel}  % deutsche Silbentrennung
\usepackage[hidelinks]{hyperref}
\usepackage{bookmark}        % besseres TOC → hyperref
\usepackage{fancyhdr}

% Versuche zuerst die gewünschte Systemschrift, sonst Fallback
\IfFontExistsTF{CaskaydiaCove Nerd Font Propo}
  {\setmainfont{CaskaydiaCove Nerd Font Propo}}
  {\setmainfont{Latin Modern Roman}} % Fallback

% Fancyhdr: Seitenzahlen rechts in Fußzeile (du kannst C->zentriert ändern)
\pagestyle{fancy}
\fancyhf{}
\fancyfoot[C]{\thepage}
\renewcommand{\headrulewidth}{0pt}
\renewcommand{\footrulewidth}{0pt}

% Metadaten für PDF
\hypersetup{
  pdftitle={Hausaufgabe 03},
  pdfauthor={Fabian Aps:525528, Joris Victor Vorderwülbecke:0528715, Emil Arthur Joseph Hartmann:052542, Friedrich Ludwig Finck:0526329},
  pdfsubject={Hausaufgabe 03 - Felix 09},
  pdfcreator={LuaLaTeX}
}

% --- Eigener Befehl für die Funktion ---
% So wird sie überall gleich angezeigt
\newcommand{\myfunc}{f(x) = e^{\cos(\ln(x))} + \sin^2(\ln(x))}

% --- ALTERNATIVER WEG (OHNE enumitem) ---
% Definiert "Level 1" der enumerate-Liste neu
\renewcommand{\labelenumi}{\alph{enumi})}
\renewcommand{\labelenumii}{\alph{enumii})} % Für verschachtelte Listen
% ----------------------------------------


\begin{document}

% ------------------
% Titelseite (keine Seitenzahl)
% ------------------
\begin{titlepage}
  \centering
  {\Large Technische Universität Berlin\par}
  \vspace{2cm}
  {\Huge\bfseries Hausaufgabe 03\par}
  \vspace{1.5cm}
  {\Large Felix 09\par}
  \vfill
  {\large Autoren: Fabian Aps:525528, Joris Victor Vorderwülbecke:0528715, Emil Arthur Joseph Hartmann:052542, Friedrich Ludwig Finck:0526329\par}
  \vspace{2cm}
  {\large \today\par}
  \thispagestyle{empty} % Titelseite ohne Seitenzahl
\end{titlepage}

% ------------------
% Inhaltsverzeichnis mit römischen Seitenzahlen (i, ii, ...)
% ------------------
\cleardoublepage
\pagenumbering{roman} % i, ii, iii, iv ...
\setcounter{page}{1}

\tableofcontents
\clearpage

% ------------------
% Hauptteil: arabische Seitenzahlen ab 1
% ------------------
\cleardoublepage
\pagenumbering{arabic}
\setcounter{page}{1}

\section{Aufgaben}\label{sec:aufgaben}

Die gegebene Funktion lautet:
$$ \myfunc $$

\begin{enumerate} % Nummerierung im Stil (a), (b), ...
    \item Für welche $x \in \mathbb{R}$ ist die Funktion definiert? \label{task:a}

    \item Zeigen Sie, dass $f\left(\frac{1}{x}\right) = f(x)$ für alle $x > 0$ gilt. \label{task:b}
    \par\textit{Hinweis: Nutzen Sie die Rechenregeln der elementaren Funktionen.}

    \item Berechnen Sie $f\left(e^{{\frac{3}{2}}\pi}\right)$. \label{task:c}
\end{enumerate}

\vspace{2cm}


\begin{figure}[h!]
    \centering
    \includegraphics[width=0.9\textwidth]{ha_03.png}
\end{figure}


% ------------------
% Lösungen
% ------------------

\cleardoublepage % Beginnt Lösungs-Teil auf einer neuen Seite

\section{Lösung zu Aufgabe a)} \label{sec:a}

\textbf{Aufgabe \ref{task:a}:} \textit{Für welche $x \in \mathbb{R}$ ist die Funktion definiert?}
\par\bigskip
Die gegebene Funktion ist:
$$ \myfunc $$
\par\bigskip
\textbf{Lösung:}
\par
Hier beginnt die Lösung für Aufgabe a)...

\clearpage
\section{Lösung zu Aufgabe b)} \label{sec:b}

\textbf{Aufgabe \ref{task:b}:} \textit{Zeigen Sie, dass $f\left(\frac{1}{x}\right) = f(x)$ für alle $x > 0$ gilt.}
\par\bigskip
Die gegebene Funktion ist:
$$ \myfunc $$
\par\bigskip
\textbf{Beweis:}
\par
Hier beginnt der Beweis für Aufgabe b)...

\clearpage
\section{Lösung zu Aufgabe c)} \label{sec:c}

\textbf{Aufgabe \ref{task:c}:} \textit{Berechnen Sie $f\left(e^{\frac{3\pi}{2}}\right)$.}
\par\bigskip
Die gegebene Funktion ist:
$$ \myfunc $$
\par\bigskip
\textbf{Berechnung:}
\par
Hier beginnt die Berechnung für Aufgabe c)...


\end{document}

% Local Variables:
% TeX-engine: luatex
% End:
