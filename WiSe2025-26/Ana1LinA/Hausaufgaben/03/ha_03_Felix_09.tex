% !TEX program = lualatex
\documentclass[12pt,a4paper]{report} % oder book

\usepackage{fontspec}        % bei lualatex/xelatex
\usepackage{microtype}
\usepackage{graphicx}
\usepackage{amsmath,amssymb}
%\usepackage{enumitem}        % Für [(a)]-Listen und \label in Listen
\usepackage[ngerman]{babel}  % deutsche Silbentrennung
\usepackage[hidelinks]{hyperref}
\usepackage{bookmark}        % besseres TOC → hyperref
\usepackage{fancyhdr}
\usepackage{nicefrac}         % Für schöne Brüche

% Versuche zuerst die gewünschte Systemschrift, sonst Fallback
\IfFontExistsTF{CaskaydiaCove Nerd Font Propo}
  {\setmainfont{CaskaydiaCove Nerd Font Propo}}
  {\setmainfont{Latin Modern Roman}} % Fallback

% Fancyhdr: Seitenzahlen rechts in Fußzeile (du kannst C->zentriert ändern)
\pagestyle{fancy}
\fancyhf{}
\fancyfoot[C]{\thepage}
\renewcommand{\headrulewidth}{0pt}
\renewcommand{\footrulewidth}{0pt}

\newcommand{\titleline}{Ana1LinA - Hausaufgabe 03}
\newcommand{\subtitleline}{Felix 09}
\newcommand{\authorsline}{Fabian Aps:525528, Joris Victor Vorderwülbecke:0528715, Emil Arthur Joseph Hartmann:052542, Friedrich Ludwig Finck:0526329}

% Metadaten für PDF
\hypersetup{
  pdftitle={\titleline},
  pdfauthor={\authorsline},
  pdfsubject={\titleline - \subtitleline},
  pdfcreator={LuaLaTeX}
}

% --- Eigener Befehl für die Funktion ---
% So wird sie überall gleich angezeigt
\newcommand{\myfunc}{f(x) = e^{\cos(\ln(x))} + \sin^2(\ln(x))}

% --- ALTERNATIVER WEG (OHNE enumitem) ---
% Definiert "Level 1" der enumerate-Liste neu
\renewcommand{\labelenumi}{\alph{enumi})}
\renewcommand{\labelenumii}{\alph{enumii})} % Für verschachtelte Listen
% ----------------------------------------


\begin{document}

% ------------------
% Titelseite (keine Seitenzahl)
% ------------------
\begin{titlepage}
  \centering
  {\Large Technische Universität Berlin\par}
  \vspace{2cm}
  {\Huge\bfseries \titleline\par}
  \vspace{1.5cm}
  {\Large \subtitleline\par}
  \vfill
  {\large Autoren: \authorsline\par}
  \vspace{2cm}
  {\large \today\par}
  \thispagestyle{empty} % Titelseite ohne Seitenzahl
\end{titlepage}

% ------------------
% Inhaltsverzeichnis mit römischen Seitenzahlen (i, ii, ...)
% ------------------
\cleardoublepage
\pagenumbering{roman} % i, ii, iii, iv ...
\setcounter{page}{1}

\tableofcontents
\clearpage

% ------------------
% Hauptteil: arabische Seitenzahlen ab 1
% ------------------
\cleardoublepage
\pagenumbering{arabic}
\setcounter{page}{1}

\chapter{Aufgaben}\label{ch:aufgaben}

Die gegebene Funktion lautet:
$$ \myfunc $$ \label{func:f}

\begin{enumerate} % Nummerierung im Stil (a), (b), ...
    \item Für welche $x \in \mathbb{R}$ ist die Funktion definiert?
\label{task:a}

    \item Zeigen Sie, dass $f\left(\frac{1}{x}\right) = f(x)$ für alle $x > 0$ gilt.
\label{task:b}
    \par\textit{Hinweis: Nutzen Sie die Rechenregeln der elementaren Funktionen.}

    \item Berechnen Sie $f\left(e^{{\frac{3}{2}}\pi}\right)$.
\label{task:c}
\end{enumerate}

\vspace{2cm}


\begin{figure}[h!]
    \centerline{\includegraphics[width=1.25\textwidth,]{ha_03.png}}
\end{figure}


% ------------------
% Lösungen
% ------------------

\cleardoublepage % Beginnt Lösungs-Teil auf einer neuen Seite

\chapter{Aufgabe a)} \label{ch:a}
\textbf{Funktion:} $ \myfunc $
\par\bigskip
\textbf{Aufgabe:} \textit{Für welche $x \in \mathbb{R}$ ist die Funktion definiert?}
\par\bigskip

\section{Lösung} \label{sec:aL}
\par
Die Funktion $f(x)$ ist definiert, solange alle ihre Teilfunktionen definiert sind. Die innerste Funktion ist der \textbf{natürliche Logarithmus} $\ln(x)$.
\begin{itemize}
    \item Der Logarithmus $\ln(x)$ ist nur für Argumente $x > 0$ definiert.
    \item Die Funktionen $\cos(z)$, $\sin^2(z)$ und $e^z$ sind für alle reellen Zahlen $z$ definiert.
\end{itemize}
Daher muss nur die Bedingung für $\ln(x)$ erfüllt sein.

\textbf{Definitionsbereich:}
$$ D = \{x \in \mathbb{R} \mid x > 0\} = (0, \infty) $$

\clearpage
\chapter{Aufgabe b)} \label{ch:b}

\textbf{Funktion:} $ \myfunc $
\par\bigskip
\textbf{Aufgabe:} \textit{Zeigen Sie, dass $f\left(\frac{1}{x}\right) = f(x)$ für alle $x > 0$ gilt.}
\par\bigskip

\section{Beweis} \label{sec:bB}
\par
Wir beginnen mit der linken Seite der Gleichung, $f\left(\frac{1}{x}\right)$, und setzen den Term in die Funktionsgleichung ein:
$$ f\left(\frac{1}{x}\right) = e^{\cos\left(\ln\left(\frac{1}{x}\right)\right)} + \sin^2\left(\ln\left(\frac{1}{x}\right)\right) $$
Nun wenden wir die Rechenregel für den Logarithmus an: $\ln\left(\frac{1}{x}\right) = -\ln(x)$ (gültig für $x>0$).
$$ f\left(\frac{1}{x}\right) = e^{\cos(-\ln(x))} + \sin^2(-\ln(x)) $$
Anschließend nutzen wir die Symmetrieeigenschaften der trigonometrischen Funktionen:
\begin{itemize}
    \item $\cos(-z) = \cos(z)$ (Kosinus ist eine gerade Funktion).
    \item $\sin(-z) = -\sin(z)$ (Sinus ist eine ungerade Funktion).
\end{itemize}
Wir setzen $z = \ln(x)$:
$$ f\left(\frac{1}{x}\right) = e^{\cos(\ln(x))} + \left(-\sin(\ln(x))\right)^2 $$
Da $\left(-\sin(\ln(x))\right)^2 = (-1)^2 \cdot \sin^2(\ln(x)) = \sin^2(\ln(x))$, erhalten wir:
$$ f\left(\frac{1}{x}\right) = e^{\cos(\ln(x))} + \sin^2(\ln(x)) $$
Dies entspricht exakt der Definition der Funktion $f(x)$:
$$ f\left(\frac{1}{x}\right) = f(x) $$
Somit ist die Behauptung gezeigt.

\clearpage
\chapter{Aufgabe c)} \label{ch:c}

\textbf{Funktion:} $ \myfunc $
\par\bigskip
\textbf{Aufgabe:} \textit{Berechnen Sie $f\left(e^{{\frac{3}{2}}\pi}\right)$.}
\par\bigskip

\section{Berechnung} \label{sec:cB}
\par
Wir setzen den Wert $x_0 = e^{{\frac{3}{2}}\pi}$ in die Funktion $f(x)$ ein:
$$ f\left(e^{\frac{3}{2}\pi}\right) = e^{\cos\left(\ln\left(e^{\frac{3}{2}\pi}\right)\right)} + \sin^2\left(\ln\left(e^{\frac{3}{2}\pi}\right)\right) $$
**Schritt 1: Vereinfachen des Logarithmus**
Mittels $\ln(e^z) = z$:
$$ \ln\left(e^{\frac{3}{2}\pi}\right) = \frac{3}{2}\pi $$
Einsetzen in die Gleichung:
$$ f\left(e^{\frac{3}{2}\pi}\right) = e^{\cos\left(\frac{3}{2}\pi\right)} + \sin^2\left(\frac{3}{2}\pi\right) $$
**Schritt 2: Einsetzen der trigonometrischen Werte**
Die trigonometrischen Werte für $\frac{3}{2}\pi$ (oder $270^\circ$) sind:
\begin{itemize}
    \item $\cos\left(\frac{3}{2}\pi\right) = 0$
    \item $\sin\left(\frac{3}{2}\pi\right) = -1$
\end{itemize}
Einsetzen in die Gleichung:
$$ f\left(e^{\frac{3}{2}\pi}\right) = e^{0} + (-1)^2 $$
**Schritt 3: Endgültige Berechnung**
Da $e^0 = 1$ und $(-1)^2 = 1$:
$$ f\left(e^{\frac{3}{2}\pi}\right) = 1 + 1 $$
$$ f\left(e^{\frac{3}{2}\pi}\right) = 2 $$


\end{document}

% Local Variables:
% TeX-engine: luatex
% End:
