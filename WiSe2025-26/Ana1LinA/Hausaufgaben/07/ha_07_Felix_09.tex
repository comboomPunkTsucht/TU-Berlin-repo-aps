% !TEX program = lualatex
\documentclass[12pt,a4paper]{report}

\usepackage{fontspec}        % bei lualatex/xelatex
\usepackage{microtype}
\usepackage{graphicx}
\usepackage{amsmath,amssymb}
\usepackage[ngerman]{babel}  % deutsche Silbentrennung
\usepackage[hidelinks]{hyperref}
\usepackage{bookmark}        % besseres TOC → hyperref
\usepackage{fancyhdr}
\usepackage{nicefrac}        % Für schöne Brüche
\usepackage{caption}         % Für Bildunterschriften
\usepackage{subcaption}      % Für Unterbilder

% Versuche zuerst die gewünschte Systemschrift, sonst Fallback
\IfFontExistsTF{CaskaydiaCove Nerd Font Propo}
  {\setmainfont{CaskaydiaCove Nerd Font Propo}}
  {\setmainfont{Latin Modern Roman}} % Fallback

% Fancyhdr: Seitenzahlen rechts in Fußzeile
\pagestyle{fancy}
\fancyhf{}
\fancyfoot[C]{\thepage}
\renewcommand{\headrulewidth}{0pt}
\renewcommand{\footrulewidth}{0pt}

\newcommand{\titleline}{Ana1LinA - Hausaufgabe 07}
\newcommand{\subtitleline}{Felix 09}
\newcommand{\authorsline}{Fabian Aps:525528, Joris Victor Vorderwülbecke:0528715, Emil Arthur Joseph Hartmann:052542, Friedrich Ludwig Finck:0526329}

% Metadaten für PDF
\hypersetup{
  pdftitle={\titleline},
  pdfauthor={\authorsline},
  pdfsubject={\titleline - \subtitleline},
  pdfcreator={LuaLaTeX}
}

\begin{document}

% ------------------
% Titelseite
% ------------------
\begin{titlepage}
  \centering
  {\Large Technische Universität Berlin\par}
  \vspace{2cm}
  {\Huge\bfseries \titleline\par}
  \vspace{1.5cm}
  {\Large \subtitleline\par}
  \vfill
  {\large Autoren: \authorsline\par}
  \vspace{2cm}
  {\large \today\par}
  \thispagestyle{empty}
\end{titlepage}

% ------------------
% Inhaltsverzeichnis
% ------------------
\cleardoublepage
\pagenumbering{roman}
\setcounter{page}{1}

\tableofcontents
\clearpage

% ------------------
% Hauptteil
% ------------------
\cleardoublepage
\pagenumbering{arabic}
\setcounter{page}{1}

\chapter{Aufgabe} \label{ch:aufgabe}

Untersuchen Sie die folgenden Funktionen auf Stetigkeit:

$$
f: \mathbb{R} \to \mathbb{R}, \quad f(x) = \begin{cases}
\cos(\pi x) + 1 & \text{für } x < 2, \\
x^2 - 1 & \text{für } x \ge 2
\end{cases}
$$

$$
g: \mathbb{R} \to \mathbb{R}, \quad g(x) = \begin{cases}
x + \frac{x+1}{|x+1|} & \text{für } x \ne -1, \\
0 & \text{für } x = -1
\end{cases}
$$

\vspace{1cm}

\begin{figure}[h!]
    \centerline{\includegraphics[width=1.0\textwidth,]{ha_07.png}}
\end{figure}

% ------------------
% Lösungen
% ------------------

\chapter{Lösung zur Aufgabe} \label{ch:loesung}

Eine Funktion $h: D \to \mathbb{R}$ heißt stetig in $a \in D$, wenn $\lim_{x \to a} h(x) = h(a)$ gilt (vgl. Definition 19.7 im Skript). Dies ist äquivalent dazu, dass der linksseitige Grenzwert $\lim_{x \nearrow a} h(x)$ und der rechtsseitige Grenzwert $\lim_{x \searrow a} h(x)$ existieren und mit dem Funktionswert $h(a)$ übereinstimmen (vgl. Abschnitt 19.2 im Skript).

\section{Untersuchung von $f$}

Die Funktion ist definiert als:
$$
f(x) = \begin{cases}
\cos(\pi x) + 1 & \text{für } x < 2 \\
x^2 - 1 & \text{für } x \ge 2
\end{cases}
$$

\subsection{Bereich $x \neq 2$}
In den Bereichen $]-\infty, 2[$ und $]2, \infty[$ ist die Funktion als Komposition und Summe elementarer, stetiger Funktionen (Kosinus, Polynome) stetig (vgl. Satz 19.10 im Skript). Der einzige kritische Punkt für die Stetigkeit ist die Nahtstelle $x_0 = 2$.

\subsection{Untersuchung an der Stelle $x_0 = 2$}
Wir berechnen den linksseitigen und den rechtsseitigen Grenzwert sowie den Funktionswert.

\textbf{1. Funktionswert:} \\
Da für $x = 2$ der Fall $x \ge 2$ gilt:
$$
f(2) = 2^2 - 1 = 4 - 1 = 3
$$

\textbf{2. Linksseitiger Grenzwert ($x \nearrow 2$):} \\
Für $x < 2$ gilt $f(x) = \cos(\pi x) + 1$. Da der Cosinus stetig ist, folgt:
$$
\lim_{x \nearrow 2} f(x) = \lim_{x \nearrow 2} (\cos(\pi x) + 1) = \cos(2\pi) + 1 = 1 + 1 = 2
$$

\textbf{3. Rechtsseitiger Grenzwert ($x \searrow 2$):} \\
Für $x > 2$ gilt $f(x) = x^2 - 1$. Da Polynome stetig sind, folgt:
$$
\lim_{x \searrow 2} f(x) = \lim_{x \searrow 2} (x^2 - 1) = 2^2 - 1 = 3
$$

\textbf{Fazit:} \\
Da der linksseitige Grenzwert ($2$) ungleich dem rechtsseitigen Grenzwert ($3$) ist, existiert der Grenzwert $\lim_{x \to 2} f(x)$ nicht.
$$
\lim_{x \nearrow 2} f(x) \neq \lim_{x \searrow 2} f(x)
$$
Damit ist die Funktion $f$ an der Stelle $x=2$ \textbf{nicht stetig}.

\section{Untersuchung von $g$}

Die Funktion ist definiert als:
$$
g(x) = \begin{cases}
x + \frac{x+1}{|x+1|} & \text{für } x \ne -1 \\
0 & \text{für } x = -1
\end{cases}
$$

\subsection{Bereich $x \neq -1$}
Für $x \neq -1$ ist die Funktion aus stetigen Funktionen (Polynome, Betrag, Quotient mit Nenner $\neq 0$) zusammengesetzt und somit stetig (vgl. Satz 19.10). Die kritische Stelle ist $x_0 = -1$.

\subsection{Untersuchung an der Stelle $x_0 = -1$}
Wir untersuchen den Term $\frac{x+1}{|x+1|}$.
$$
\frac{x+1}{|x+1|} = \begin{cases}
\frac{x+1}{x+1} = 1 & \text{für } x+1 > 0 \iff x > -1 \\
\frac{x+1}{-(x+1)} = -1 & \text{für } x+1 < 0 \iff x < -1
\end{cases}
$$

\textbf{1. Funktionswert:} \\
Laut Definition ist $g(-1) = 0$.

\textbf{2. Linksseitiger Grenzwert ($x \nearrow -1$):} \\
Für $x < -1$ ist $\frac{x+1}{|x+1|} = -1$.
$$
\lim_{x \nearrow -1} g(x) = \lim_{x \nearrow -1} (x - 1) = -1 - 1 = -2
$$

\textbf{3. Rechtsseitiger Grenzwert ($x \searrow -1$):} \\
Für $x > -1$ ist $\frac{x+1}{|x+1|} = 1$.
$$
\lim_{x \searrow -1} g(x) = \lim_{x \searrow -1} (x + 1) = -1 + 1 = 0
$$

\textbf{Fazit:} \\
Der linksseitige Grenzwert ($-2$) ist ungleich dem rechtsseitigen Grenzwert ($0$). Somit existiert der Grenzwert $\lim_{x \to -1} g(x)$ nicht.
Zwar stimmt der rechtsseitige Grenzwert mit dem Funktionswert überein ($0 = g(-1)$), für die Stetigkeit müssen jedoch beide einseitigen Grenzwerte mit dem Funktionswert übereinstimmen.
Daher ist $g$ an der Stelle $x=-1$ \textbf{nicht stetig}.

\end{document}

% Local Variables:
% TeX-engine: luatex
% End:
