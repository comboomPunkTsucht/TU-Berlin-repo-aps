% !TEX program = lualatex
\documentclass[12pt,a4paper]{report} % oder book

\usepackage{fontspec}        % bei lualatex/xelatex
\usepackage{microtype}
\usepackage{graphicx}
\usepackage{amsmath,amssymb}
\usepackage[ngerman]{babel}  % deutsche Silbentrennung
\usepackage[hidelinks]{hyperref}
\usepackage{bookmark}        % besseres TOC → hyperref
\usepackage{fancyhdr}
\usepackage{nicefrac}         % Für schöne Brüche

% Versuche zuerst die gewünschte Systemschrift, sonst Fallback
\IfFontExistsTF{CaskaydiaCove Nerd Font Propo}
  {\setmainfont{CaskaydiaCove Nerd Font Propo}}
  {\setmainfont{Latin Modern Roman}} % Fallback

% Fancyhdr: Seitenzahlen rechts in Fußzeile (du kannst C->zentriert ändern)
\pagestyle{fancy}
\fancyhf{}
\fancyfoot[C]{\thepage}
\renewcommand{\headrulewidth}{0pt}
\renewcommand{\footrulewidth}{0pt}

\newcommand{\titleline}{Ana1LinA - Hausaufgabe 04}
\newcommand{\subtitleline}{Felix 09}
\newcommand{\authorsline}{Fabian Aps:525528, Joris Victor Vorderwülbecke:0528715, Emil Arthur
Joseph Hartmann:052542, Friedrich Ludwig Finck:0526329}

% Metadaten für PDF
\hypersetup{
  pdftitle={\titleline},
  pdfauthor={\authorsline},
  pdfsubject={\titleline - \subtitleline},
  pdfcreator={LuaLaTeX}
}

% --- Eigener Befehl für die Funktion / das Polynom ---
% Anpassung an die neue Aufgabe
\newcommand{\mypolynom}{p_a(x) = x^3 + (a^2-2a+1)x^2 + (-2a^3+a^2-2a)x - 2a^3}

% --- ALTERNATIVER WEG (OHNE enumitem) ---
% Definiert "Level 1" der enumerate-Liste neu
\renewcommand{\labelenumi}{\alph{enumi})}
\renewcommand{\labelenumii}{\alph{enumii})} % Für verschachtelte Listen
% ----------------------------------------


\begin{document}

% ------------------
% Titelseite (keine Seitenzahl)
% ------------------
\begin{titlepage}
  \centering
  {\Large Technische Universität Berlin\par}
  \vspace{2cm}
  {\Huge\bfseries \titleline\par}
  \vspace{1.5cm}
  {\Large \subtitleline\par}
  \vfill
  {\large Autoren: \authorsline\par}
  \vspace{2cm}
  {\large \today\par}
  \thispagestyle{empty} % Titelseite ohne Seitenzahl
\end{titlepage}

% ------------------
% Inhaltsverzeichnis mit römischen Seitenzahlen (i, ii, ...)
% ------------------
\cleardoublepage
\pagenumbering{roman}
\setcounter{page}{1}

\tableofcontents
\clearpage

% ------------------
% Hauptteil: arabische Seitenzahlen ab 1
% ------------------
\cleardoublepage
\pagenumbering{arabic}
\setcounter{page}{1}

\chapter{Aufgaben}\label{ch:aufgaben}

Das gegebene Polynom $p_a(x)$ für $a \in \mathbb{R}$ lautet:
$$ \mypolynom $$ \label{func:p}

\begin{enumerate} % Nummerierung im Stil (a), (b), ...
    \item Berechnen Sie $p_a(2a)$.
    \item Bestimmen Sie alle Nullstellen von $p_a$.
    \item Berechnen Sie die komplexe Linearfaktorzerlegung von $p_a$.
    \item Berechnen Sie die reelle Zerlegung von $p_a$.
\end{enumerate}

\vspace{2cm}


\begin{figure}[h!]
    \centerline{\includegraphics[width=1.25\textwidth,]{ha_04.png}}
\end{figure}


% ------------------
% Lösungen
% ------------------

\cleardoublepage % Beginnt Lösungs-Teil auf einer neuen Seite

\chapter{Aufgabe a)} \label{ch:a}
\textbf{Polynom:} $ p_a(x) = x^3 + (a^2-2a+1)x^2 + (-2a^3+a^2-2a)x - 2a^3 $
\par\bigskip
\textbf{Aufgabe:} \textit{Berechnen Sie $p_a(2a)$.}
\par\bigskip

\section{Berechnung} \label{sec:aL}
\par
Wir setzen $x = 2a$ in das Polynom $p_a(x)$ ein:
$$ p_a(2a) = (2a)^3 + (a^2-2a+1)(2a)^2 + (-2a^3+a^2-2a)(2a) - 2a^3 $$
Zuerst vereinfachen wir die Terme mit $(2a)$:
\begin{align*}
p_a(2a) &= 8a^3 + (a^2-2a+1)(4a^2) + (-2a^3+a^2-2a)(2a) - 2a^3 \\
\end{align*}
Nun multiplizieren wir die Produkte aus:
\begin{align*}
p_a(2a) &= 8a^3 + (4a^4 - 8a^3 + 4a^2) + (-4a^4 + 2a^3 - 4a^2) - 2a^3 \\
\end{align*}
Wir fassen die Terme zusammen, indem wir nach Potenzen von $a$ ordnen:
$$ p_a(2a) = (4a^4 - 4a^4) + (8a^3 - 8a^3 + 2a^3 - 2a^3) + (4a^2 - 4a^2) $$
Alle Terme heben sich gegenseitig auf:
$$ p_a(2a) = 0 $$
\par
\textbf{Ergebnis:} $p_a(2a) = 0$. Dies bedeutet, dass $x_1 = 2a$ eine Nullstelle des Polynoms ist.


\clearpage
\chapter{Aufgabe b), c) und d)} \label{ch:bcd}

\textbf{Polynom:} $ p_a(x) = x^3 + (a^2-2a+1)x^2 + (-2a^3+a^2-2a)x - 2a^3 $
\par\bigskip
\textbf{Aufgabe:} \textit{Bestimmen Sie alle Nullstellen von $p_a$. Berechnen Sie die komplexe und reelle Zerlegung von $p_a$.}
\par\bigskip

\section{Nullstellenbestimmung (Aufgabe b))} \label{sec:bL}
\par
Aus Aufgabe a) wissen wir, dass $x_1 = 2a$ eine Nullstelle ist, da $p_a(2a) = 0$.
Wir können nun eine \textbf{Polynomdivision} durch den Linearfaktor $(x - 2a)$ durchführen.
$$ \left(x^3 + (a^2-2a+1)x^2 + (-2a^3+a^2-2a)x - 2a^3\right) : (x - 2a) $$

\par
\begin{center}
\begin{tabular}{r l}
  & $\left(x^3 + (a^2-2a+1)x^2 + (-2a^3+a^2-2a)x - 2a^3\right) : (x - 2a) = x^2 + (a^2+1)x + a^2$ \\
$-$ & $\underline{(x^3 - 2ax^2)}$ \\
  & $(a^2+1)x^2 + (-2a^3+a^2-2a)x$ \\
$-$ & $\underline{((a^2+1)x^2 - 2a(a^2+1)x)}$ \\
  & $( -2a^3+a^2-2a + 2a^3+2a )x - 2a^3$ \\
  & $( a^2 )x - 2a^3$ \\
$-$ & $\underline{(a^2 x - 2a^3)}$ \\
  & $0$
\end{tabular}
\end{center}
\par
Der Quotient ist das quadratische Polynom $q(x) = x^2 + (a^2+1)x + a^2$. Die verbleibenden Nullstellen $x_{2,3}$ erhalten wir durch Anwenden der Mitternachtsformel (p-q-Formel oder abc-Formel) auf $q(x) = 0$.

$$ x^2 + (a^2+1)x + a^2 = 0 $$
Mit $A=1$, $B=(a^2+1)$ und $C=a^2$:
\begin{align*}
x_{2,3} &= \frac{-B \pm \sqrt{B^2 - 4AC}}{2A} \\
x_{2,3} &= \frac{-(a^2+1) \pm \sqrt{(a^2+1)^2 - 4(1)(a^2)}}{2} \\
x_{2,3} &= \frac{-(a^2+1) \pm \sqrt{a^4 + 2a^2 + 1 - 4a^2}}{2} \\
x_{2,3} &= \frac{-(a^2+1) \pm \sqrt{a^4 - 2a^2 + 1}}{2} \\
\end{align*}
Der Ausdruck unter der Wurzel ist ein vollständiges Quadrat: $a^4 - 2a^2 + 1 = (a^2 - 1)^2$.
\begin{align*}
x_{2,3} &= \frac{-(a^2+1) \pm \sqrt{(a^2 - 1)^2}}{2} \\
x_{2,3} &= \frac{-(a^2+1) \pm |a^2 - 1|}{2} \\
\end{align*}
Wir betrachten die zwei Fälle für das $\pm$:
\par
\textbf{Fall 1: Positives Vorzeichen ($+$)}
$$ x_2 = \frac{-(a^2+1) + (a^2 - 1)}{2} = \frac{-a^2 - 1 + a^2 - 1}{2} = \frac{-2}{2} = -1 $$
\par
\textbf{Fall 2: Negatives Vorzeichen ($-$) - muss beachtet werden, da $|a^2 - 1| = -(a^2-1)$ möglich wäre.}
Unabhängig davon, ob wir $\sqrt{(a^2-1)^2} = a^2-1$ oder $-(a^2-1)$ annehmen, führt der zweite Fall in der $\pm$ Rechnung zum gleichen Ergebnis:
$$ x_3 = \frac{-(a^2+1) - (a^2 - 1)}{2} = \frac{-a^2 - 1 - a^2 + 1}{2} = \frac{-2a^2}{2} = -a^2 $$
\par
\textbf{Alle Nullstellen von $p_a$ (Aufgabe b)):}
$$ \mathcal{N} = \{2a, -1, -a^2\} $$


\section{Komplexe Linearfaktorzerlegung (Aufgabe c))} \label{sec:cL}
\par
Die Linearfaktorzerlegung eines normierten Polynoms dritten Grades $p(x)$ mit den Nullstellen $x_1, x_2, x_3$ lautet: $p(x) = (x-x_1)(x-x_2)(x-x_3)$. Da alle Nullstellen reell sind (für $a \in \mathbb{R}$), ist die komplexe Zerlegung identisch mit der reellen.
$$ p_a(x) = (x - 2a)(x - (-1))(x - (-a^2)) $$
$$ p_a(x) = (x - 2a)(x + 1)(x + a^2) $$
Diese Zerlegung gilt für alle $x \in \mathbb{C}$.

\section{Reelle Zerlegung (Aufgabe d))} \label{sec:dL}
\par
Da alle Nullstellen $x_1=2a$, $x_2=-1$ und $x_3=-a^2$ reell sind, besteht die reelle Zerlegung nur aus Linearfaktoren und ist somit identisch mit der komplexen Linearfaktorzerlegung.
$$ p_a(x) = (x - 2a)(x + 1)(x + a^2) $$


\end{document}
