% !TEX program = lualatex
\documentclass[12pt,a4paper]{report}

\usepackage{fontspec}        % bei lualatex/xelatex
\usepackage{microtype}
\usepackage{graphicx}
\usepackage{amsmath,amssymb}
\usepackage[ngerman]{babel}  % deutsche Silbentrennung
\usepackage[hidelinks]{hyperref}
\usepackage{bookmark}        % besseres TOC → hyperref
\usepackage{fancyhdr}
\usepackage{nicefrac}        % Für schöne Brüche
\usepackage{caption}         % Für Bildunterschriften
\usepackage{subcaption}      % Für Unterbilder

% Versuche zuerst die gewünschte Systemschrift, sonst Fallback
\IfFontExistsTF{CaskaydiaCove Nerd Font Propo}
  {\setmainfont{CaskaydiaCove Nerd Font Propo}}
  {\setmainfont{Latin Modern Roman}} % Fallback

% Fancyhdr: Seitenzahlen rechts in Fußzeile
\pagestyle{fancy}
\fancyhf{}
\fancyfoot[C]{\thepage}
\renewcommand{\headrulewidth}{0pt}
\renewcommand{\footrulewidth}{0pt}

\newcommand{\titleline}{Ana1LinA - Hausaufgabe 06}
\newcommand{\subtitleline}{Felix 09}
\newcommand{\authorsline}{Fabian Aps:525528, Joris Victor Vorderwülbecke:0528715, Emil Arthur Joseph Hartmann:052542, Friedrich Ludwig Finck:0526329}

% Metadaten für PDF
\hypersetup{
  pdftitle={\titleline},
  pdfauthor={\authorsline},
  pdfsubject={\titleline - \subtitleline},
  pdfcreator={LuaLaTeX}
}

% Eigener Befehl für die elementaren Zeilenoperationen (falls benötigt)
\newcommand{\ZOp}[2]{\text{Z}_{#1} \leftarrow #2}

% --- ALTERNATIVER WEG (OHNE enumitem) ---
\renewcommand{\labelenumi}{\alph{enumi})}
\renewcommand{\labelenumii}{\alph{enumii})}

\begin{document}

% ------------------
% Titelseite
% ------------------
\begin{titlepage}
  \centering
  {\Large Technische Universität Berlin\par}
  \vspace{2cm}
  {\Huge\bfseries \titleline\par}
  \vspace{1.5cm}
  {\Large \subtitleline\par}
  \vfill
  {\large Autoren: \authorsline\par}
  \vspace{2cm}
  {\large \today\par}
  \thispagestyle{empty}
\end{titlepage}

% ------------------
% Inhaltsverzeichnis
% ------------------
\cleardoublepage
\pagenumbering{roman}
\setcounter{page}{1}

\tableofcontents
\clearpage

% ------------------
% Hauptteil
% ------------------
\cleardoublepage
\pagenumbering{arabic}
\setcounter{page}{1}

\chapter{Aufgabe 11} \label{ch:aufgabe}

Gegeben sei die Abbildung $f : \mathbb{C}^3 \to \mathbb{C}^2$, definiert durch:
$$
\begin{bmatrix} a \\ b \\ c \end{bmatrix} \mapsto \begin{bmatrix} ia + b \\ 2ic \end{bmatrix}
$$

Es sind folgende Teilaufgaben zu bearbeiten:
\begin{enumerate}
    \item Zeigen Sie, dass $f$ linear ist.
    \item Finden Sie eine Matrix $A$, sodass $f\left(\begin{bmatrix} a \\ b \\ c \end{bmatrix}\right) = A \begin{bmatrix} a \\ b \\ c \end{bmatrix}$ für alle $\begin{bmatrix} a \\ b \\ c \end{bmatrix} \in \mathbb{C}^3$ gilt.
    \item Bestimmen Sie $\text{Kern}(f)$ und geben Sie eine Basis des Kerns an.
    \item Ist $f$ injektiv/surjektiv?
\end{enumerate}

%\vspace{2cm}


\begin{figure}[h!]
    \centerline{\includegraphics[width=1.25\textwidth,]{ha_06.png}}
\end{figure}

% ------------------
% Lösungen
% ------------------

\chapter{Lösung zu Aufgabe 11} \label{ch:loesung}

\section{a) Linearität von $f$}
Zu zeigen ist, dass für alle $\vec{u}, \vec{v} \in \mathbb{C}^3$ und alle $\lambda \in \mathbb{C}$ gilt:
\begin{enumerate}
    \item $f(\vec{u} + \vec{v}) = f(\vec{u}) + f(\vec{v})$ (Additivität)
    \item $f(\lambda \vec{u}) = \lambda f(\vec{u})$ (Homogenität)
\end{enumerate}

Seien $\vec{u} = \begin{bmatrix} a_1 \\ b_1 \\ c_1 \end{bmatrix}$ und $\vec{v} = \begin{bmatrix} a_2 \\ b_2 \\ c_2 \end{bmatrix}$.

\textbf{Additivität:}
\begin{align*}
f(\vec{u} + \vec{v}) &= f\left( \begin{bmatrix} a_1+a_2 \\ b_1+b_2 \\ c_1+c_2 \end{bmatrix} \right)
= \begin{bmatrix} i(a_1+a_2) + (b_1+b_2) \\ 2i(c_1+c_2) \end{bmatrix} \\
&= \begin{bmatrix} (ia_1+b_1) + (ia_2+b_2) \\ 2ic_1 + 2ic_2 \end{bmatrix}
= \begin{bmatrix} ia_1+b_1 \\ 2ic_1 \end{bmatrix} + \begin{bmatrix} ia_2+b_2 \\ 2ic_2 \end{bmatrix} \\
&= f(\vec{u}) + f(\vec{v})
\end{align*}

\textbf{Homogenität:}
\begin{align*}
f(\lambda \vec{u}) &= f\left( \begin{bmatrix} \lambda a_1 \\ \lambda b_1 \\ \lambda c_1 \end{bmatrix} \right)
= \begin{bmatrix} i(\lambda a_1) + (\lambda b_1) \\ 2i(\lambda c_1) \end{bmatrix} \\
&= \begin{bmatrix} \lambda(ia_1 + b_1) \\ \lambda(2ic_1) \end{bmatrix}
= \lambda \begin{bmatrix} ia_1 + b_1 \\ 2ic_1 \end{bmatrix} \\
&= \lambda f(\vec{u})
\end{align*}
Da beide Eigenschaften erfüllt sind, ist $f$ linear.

\section{b) Matrixdarstellung $A$}
Wir suchen eine Matrix $A \in \mathbb{C}^{2 \times 3}$, sodass gilt:
$$
A \begin{bmatrix} a \\ b \\ c \end{bmatrix} = \begin{bmatrix} ia + b \\ 2ic \end{bmatrix}
$$
Wir können den Ergebnisvektor als Linearkombination der Eingabekomponenten schreiben:
$$
\begin{bmatrix} ia + b \\ 2ic \end{bmatrix} = \begin{bmatrix} i \cdot a + 1 \cdot b + 0 \cdot c \\ 0 \cdot a + 0 \cdot b + 2i \cdot c \end{bmatrix}
$$
Daraus lassen sich die Koeffizienten direkt in die Matrix $A$ übertragen:
$$
A = \begin{bmatrix}
i & 1 & 0 \\
0 & 0 & 2i
\end{bmatrix}
$$

\section{c) Kern von $f$}
Der Kern von $f$ ist die Menge aller Vektoren $\vec{x} \in \mathbb{C}^3$, die auf den Nullvektor abgebildet werden, also $A\vec{x} = \vec{0}$.
Wir lösen das homogene lineare Gleichungssystem:
$$
\begin{bmatrix}
i & 1 & 0 \\
0 & 0 & 2i
\end{bmatrix}
\begin{bmatrix} a \\ b \\ c \end{bmatrix}
=
\begin{bmatrix} 0 \\ 0 \end{bmatrix}
$$
Daraus ergeben sich zwei Gleichungen:
\begin{enumerate}
    \item $ia + b = 0 \implies b = -ia$
    \item $2ic = 0 \implies c = 0$ (da $2i \neq 0$)
\end{enumerate}
Die Variable $a$ ist frei wählbar. Setzen wir $a = t$ mit $t \in \mathbb{C}$, so folgt $b = -it$.
Der Lösungsvektor lautet somit:
$$
\vec{x} = \begin{bmatrix} t \\ -it \\ 0 \end{bmatrix} = t \begin{bmatrix} 1 \\ -i \\ 0 \end{bmatrix}
$$
Der Kern ist also:
$$
\text{Kern}(f) = \text{span}\left( \begin{bmatrix} 1 \\ -i \\ 0 \end{bmatrix} \right)
$$
Eine Basis des Kerns ist:
$$
\mathcal{B}_{\text{Kern}} = \left\{ \begin{bmatrix} 1 \\ -i \\ 0 \end{bmatrix} \right\}
$$

\section{d) Injektivität und Surjektivität}

\textbf{Injektivität:} \\
Eine lineare Abbildung ist genau dann injektiv, wenn ihr Kern nur den Nullvektor enthält (trivialer Kern).
Aus Teilaufgabe c) wissen wir, dass $\text{Kern}(f) \neq \{\vec{0}\}$, da z.B. $\begin{bmatrix} 1 \\ -i \\ 0 \end{bmatrix} \in \text{Kern}(f)$.
Daraus folgt: $f$ ist \textbf{nicht injektiv}.

\textbf{Surjektivität:} \\
Eine lineare Abbildung $f: V \to W$ ist surjektiv, wenn $\text{Bild}(f) = W$. Alternativ gilt der Dimensionssatz (Rangformel):
$$
\dim(V) = \dim(\text{Kern}(f)) + \dim(\text{Bild}(f))
$$
Hier ist $V = \mathbb{C}^3$, also $\dim(V) = 3$. Aus c) wissen wir, dass $\dim(\text{Kern}(f)) = 1$ (da die Basis ein Element hat).
$$
3 = 1 + \dim(\text{Bild}(f)) \implies \dim(\text{Bild}(f)) = 2
$$
Der Zielraum ist $W = \mathbb{C}^2$, welcher ebenfalls die Dimension 2 hat. Da die Dimension des Bildes gleich der Dimension des Zielraums ist, muss das Bild der gesamte Zielraum sein.
Daraus folgt: $f$ ist \textbf{surjektiv}.

\end{document}

% Local Variables:
% TeX-engine: luatex
% End:
