% !TEX program = lualatex
\documentclass[12pt,a4paper]{report} % oder book

\usepackage{fontspec}        % bei lualatex/xelatex
\usepackage{microtype}
\usepackage{graphicx}
\usepackage{amsmath,amssymb}
\usepackage[ngerman]{babel}  % deutsche Silbentrennung
\usepackage[hidelinks]{hyperref}
\usepackage{bookmark}        % besseres TOC → hyperref
\usepackage{fancyhdr}
\usepackage{nicefrac}         % Für schöne Brüche
\usepackage{caption}          % Für Bildunterschriften
\usepackage{subcaption}       % Für Unterbilder (nicht benötigt, aber gut)

% Versuche zuerst die gewünschte Systemschrift, sonst Fallback
\IfFontExistsTF{CaskaydiaCove Nerd Font Propo}
  {\setmainfont{CaskaydiaCove Nerd Font Propo}}
  {\setmainfont{Latin Modern Roman}} % Fallback

% Fancyhdr: Seitenzahlen rechts in Fußzeile
\pagestyle{fancy}
\fancyhf{}
\fancyfoot[C]{\thepage}
\renewcommand{\headrulewidth}{0pt}
\renewcommand{\footrulewidth}{0pt}

\newcommand{\titleline}{Ana1LinA - Hausaufgabe 05}
\newcommand{\subtitleline}{Felix 09}
\newcommand{\authorsline}{Fabian Aps:525528, Joris Victor Vorderwülbecke:0528715, Emil Arthur
Joseph Hartmann:052542, Friedrich Ludwig Finck:0526329}

% Metadaten für PDF
\hypersetup{
  pdftitle={\titleline},
  pdfauthor={\authorsline},
  pdfsubject={\titleline - \subtitleline},
  pdfcreator={LuaLaTeX}
}

% Eigener Befehl für die elementaren Zeilenoperationen
\newcommand{\ZOp}[2]{\text{Z}_{#1} \leftarrow #2}

% --- ALTERNATIVER WEG (OHNE enumitem) ---
% Definiert "Level 1" der enumerate-Liste neu
\renewcommand{\labelenumi}{\alph{enumi})}
\renewcommand{\labelenumii}{\alph{enumii})} % Für verschachtelte Listen
% ----------------------------------------


\begin{document}

% ------------------
% Titelseite (keine Seitenzahl)
% ------------------
\begin{titlepage}
  \centering
  {\Large Technische Universität Berlin\par}
  \vspace{2cm}
  {\Huge\bfseries \titleline\par}
  \vspace{1.5cm}
  {\Large \subtitleline\par}
  \vfill
  {\large Autoren: \authorsline\par}
  \vspace{2cm}
  {\large \today\par}
  \thispagestyle{empty} % Titelseite ohne Seitenzahl
\end{titlepage}

% ------------------
% Inhaltsverzeichnis mit römischen Seitenzahlen (i, ii, ...)
% ------------------
\cleardoublepage
\pagenumbering{roman}
\setcounter{page}{1}

\tableofcontents
\clearpage

% ------------------
% Hauptteil: arabische Seitenzahlen ab 1
% ------------------
\cleardoublepage
\pagenumbering{arabic}
\setcounter{page}{1}

\chapter{Aufgabenstellung}\label{ch:aufgaben}

Die Hausaufgabe verlangt die Lösung der folgenden reellen linearen Gleichungssysteme, indem die erweiterte Koeffizientenmatrix auf \textbf{normierte Zeilenstufenform} gebracht wird. Die verwendeten elementaren Zeilenoperationen sind zu kennzeichnen und die Lösungsmenge ist anzugeben.

\begin{enumerate}
    \item $$ \begin{bmatrix} 2 & 3 & -1 \\ 1 & -2 & 1 \\ -1 & 1 & 2 \end{bmatrix} \begin{bmatrix} x_1 \\ x_2 \\ x_3 \end{bmatrix} = \begin{bmatrix} -9 \\ 9 \\ 0 \end{bmatrix} $$
    \item $$ \begin{bmatrix} 1 & 4 & 5 \\ 1 & 0 & 1 \\ 0 & 4 & 4 \end{bmatrix} \begin{bmatrix} x_1 \\ x_2 \\ x_3 \end{bmatrix} = \begin{bmatrix} 18 \\ 2 \\ 16 \end{bmatrix} $$
\end{enumerate}

\begin{figure}[h!]
    \centering
    \includegraphics[width=1.0\textwidth]{ha_05.png}
    \caption{Abbildung der Aufgabenstellung.}
\end{figure}

% ------------------
% Lösungen
% ------------------

\cleardoublepage % Beginnt Lösungs-Teil auf einer neuen Seite

\chapter{Lösung des linearen Gleichungssystems a)} \label{ch:a}

Das Gleichungssystem a) lautet:
$$ \begin{bmatrix} 2 & 3 & -1 \\ 1 & -2 & 1 \\ -1 & 1 & 2 \end{bmatrix} \begin{bmatrix} x_1 \\ x_2 \\ x_3 \end{bmatrix} = \begin{bmatrix} -9 \\ 9 \\ 0 \end{bmatrix} $$

Wir verwenden das Gauß-Jordan-Verfahren, um die erweiterte Koeffizientenmatrix $[A|\vec{b}]$ in die normierte Zeilenstufenform zu überführen.

\section{Überführung in die normierte Zeilenstufenform} \label{sec:aL}

Die erweiterte Koeffizientenmatrix ist:
$$
\begin{bmatrix}
2 & 3 & -1 & | & -9 \\
1 & -2 & 1 & | & 9 \\
-1 & 1 & 2 & | & 0
\end{bmatrix}
$$

\begin{align*}
\begin{bmatrix}
2 & 3 & -1 & | & -9 \\
1 & -2 & 1 & | & 9 \\
-1 & 1 & 2 & | & 0
\end{bmatrix}
& \xrightarrow{\text{Z}_1 \leftrightarrow \text{Z}_2}
\begin{bmatrix}
\mathbf{1} & -2 & 1 & | & 9 \\
2 & 3 & -1 & | & -9 \\
-1 & 1 & 2 & | & 0
\end{bmatrix}
\\
& \xrightarrow{\substack{\ZOp{2}{\text{Z}_2 - 2 \cdot \text{Z}_1} \\ \ZOp{3}{\text{Z}_3 + \text{Z}_1}}}
\begin{bmatrix}
1 & -2 & 1 & | & 9 \\
0 & \mathbf{7} & -3 & | & -27 \\
0 & -1 & 3 & | & 9
\end{bmatrix}
\\
& \xrightarrow{\text{Z}_2 \leftrightarrow \text{Z}_3}
\begin{bmatrix}
1 & -2 & 1 & | & 9 \\
0 & -1 & 3 & | & 9 \\
0 & 7 & -3 & | & -27
\end{bmatrix}
\\
& \xrightarrow{\substack{\ZOp{2}{-1 \cdot \text{Z}_2} \\ \ZOp{3}{\text{Z}_3 - 7 \cdot \text{Z}_2}}}
\begin{bmatrix}
1 & -2 & 1 & | & 9 \\
0 & \mathbf{1} & -3 & | & -9 \\
0 & 0 & 18 & | & 36
\end{bmatrix}
\\
& \xrightarrow{\ZOp{3}{\frac{1}{18} \cdot \text{Z}_3}}
\begin{bmatrix}
1 & -2 & 1 & | & 9 \\
0 & 1 & -3 & | & -9 \\
0 & 0 & \mathbf{1} & | & 2
\end{bmatrix}
\\
& \xrightarrow{\substack{\ZOp{1}{\text{Z}_1 + 2 \cdot \text{Z}_2} \\ \ZOp{2}{\text{Z}_2 + 3 \cdot \text{Z}_3}}}
\begin{bmatrix}
1 & 0 & -5 & | & -9 \\
0 & 1 & 0 & | & -3 \\
0 & 0 & 1 & | & 2
\end{bmatrix}
\\
& \xrightarrow{\ZOp{1}{\text{Z}_1 + 5 \cdot \text{Z}_3}}
\begin{bmatrix}
\mathbf{1} & 0 & 0 & | & 1 \\
0 & \mathbf{1} & 0 & | & -3 \\
0 & 0 & \mathbf{1} & | & 2
\end{bmatrix}
\end{align*}

\section{Lösung und Lösungsmenge} \label{sec:a_ergebnis}
Die normierte Zeilenstufenform liefert die eindeutige Lösung:
$$ x_1 = 1, \quad x_2 = -3, \quad x_3 = 2 $$
Der Rang der Koeffizientenmatrix ist $\text{Rang}(A) = 3$, und der Rang der erweiterten Koeffizientenmatrix ist $\text{Rang}([A|\vec{b}]) = 3$. Da dies der Anzahl der Variablen entspricht, existiert eine eindeutige Lösung.

Die Lösungsmenge $\mathcal{L}_a$ ist:
$$ \mathcal{L}_a = \left\{ \begin{bmatrix} 1 \\ -3 \\ 2 \end{bmatrix} \right\} $$

\clearpage
\chapter{Lösung des linearen Gleichungssystems b)} \label{ch:b}

Das Gleichungssystem b) lautet:
$$ \begin{bmatrix} 1 & 4 & 5 \\ 1 & 0 & 1 \\ 0 & 4 & 4 \end{bmatrix} \begin{bmatrix} x_1 \\ x_2 \\ x_3 \end{bmatrix} = \begin{bmatrix} 18 \\ 2 \\ 16 \end{bmatrix} $$

\section{Überführung in die normierte Zeilenstufenform} \label{sec:bL}

Die erweiterte Koeffizientenmatrix ist:
$$
\begin{bmatrix}
1 & 4 & 5 & | & 18 \\
1 & 0 & 1 & | & 2 \\
0 & 4 & 4 & | & 16
\end{bmatrix}
$$

\begin{align*}
\begin{bmatrix}
\mathbf{1} & 4 & 5 & | & 18 \\
1 & 0 & 1 & | & 2 \\
0 & 4 & 4 & | & 16
\end{bmatrix}
& \xrightarrow{\ZOp{2}{\text{Z}_2 - \text{Z}_1}}
\begin{bmatrix}
1 & 4 & 5 & | & 18 \\
0 & -4 & -4 & | & -16 \\
0 & 4 & 4 & | & 16
\end{bmatrix}
\\
& \xrightarrow{\ZOp{2}{-\frac{1}{4} \cdot \text{Z}_2}}
\begin{bmatrix}
1 & 4 & 5 & | & 18 \\
0 & \mathbf{1} & 1 & | & 4 \\
0 & 4 & 4 & | & 16
\end{bmatrix}
\\
& \xrightarrow{\substack{\ZOp{1}{\text{Z}_1 - 4 \cdot \text{Z}_2} \\ \ZOp{3}{\text{Z}_3 - 4 \cdot \text{Z}_2}}}
\begin{bmatrix}
\mathbf{1} & 0 & 1 & | & 2 \\
0 & \mathbf{1} & 1 & | & 4 \\
0 & 0 & 0 & | & 0
\end{bmatrix}
\end{align*}
Die resultierende Matrix ist in der normierten Zeilenstufenform.

\section{Lösung und Lösungsmenge} \label{sec:b_ergebnis}

Die normierte Zeilenstufenform entspricht dem folgenden Gleichungssystem:
$$
\begin{aligned}
x_1 + x_3 &= 2 \\
x_2 + x_3 &= 4 \\
0 &= 0
\end{aligned}
$$
Der Rang der Koeffizientenmatrix und der erweiterten Koeffizientenmatrix ist $\text{Rang}(A) = \text{Rang}([A|\vec{b}]) = 2$. Da dieser Rang kleiner als die Anzahl der Variablen ($n=3$) ist, existieren unendlich viele Lösungen.

Wir wählen $x_3$ als freien Parameter $t \in \mathbb{R}$:
$$ x_3 = t $$
Wir lösen nach den Basisvariablen $x_1$ und $x_2$ auf:
$$
\begin{aligned}
x_1 &= 2 - x_3 & \implies x_1 &= 2 - t \\
x_2 &= 4 - x_3 & \implies x_2 &= 4 - t
\end{aligned}
$$
Der Lösungsvektor $\vec{x}$ ist:
$$ \vec{x} = \begin{bmatrix} x_1 \\ x_2 \\ x_3 \end{bmatrix} = \begin{bmatrix} 2 - t \\ 4 - t \\ t \end{bmatrix} = \begin{bmatrix} 2 \\ 4 \\ 0 \end{bmatrix} + t \begin{bmatrix} -1 \\ -1 \\ 1 \end{bmatrix} $$

Die Lösungsmenge $\mathcal{L}_b$ ist:
$$ \mathcal{L}_b = \left\{ \begin{bmatrix} 2 \\ 4 \\ 0 \end{bmatrix} + t \begin{bmatrix} -1 \\ -1 \\ 1 \end{bmatrix} \mid t \in \mathbb{R} \right\} $$

\end{document}

% Local Variables:
% TeX-engine: luatex
% End:
