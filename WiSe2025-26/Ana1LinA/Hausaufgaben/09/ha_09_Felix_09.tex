% !TEX program = lualatex
\documentclass[12pt,a4paper]{report}

\usepackage{fontspec}
\usepackage{microtype}
\usepackage{graphicx}
\usepackage{amsmath,amssymb}
\usepackage[ngerman]{babel}
\usepackage[hidelinks]{hyperref}
\usepackage{bookmark}
\usepackage{fancyhdr}
\usepackage{nicefrac}
\usepackage{caption}
\usepackage{subcaption}

% Seitenränder anpassen
\usepackage[a4paper, margin=2.5cm]{geometry}

% Schriftart-Einstellungen
\IfFontExistsTF{CaskaydiaCove Nerd Font Propo}
  {\setmainfont{CaskaydiaCove Nerd Font Propo}}
  {\setmainfont{Latin Modern Roman}}

% Kopf- und Fußzeilen
\pagestyle{fancy}
\fancyhf{}
\fancyfoot[C]{\thepage}
\renewcommand{\headrulewidth}{0pt}
\renewcommand{\footrulewidth}{0pt}

% Metadaten
\newcommand{\titleline}{Ana1LinA - Hausaufgabe 09}
\newcommand{\subtitleline}{Felix 09}
\newcommand{\authorsline}{Fabian Aps:525528, Joris Victor Vorderwülbecke:0528715, Emil Arthur Joseph Hartmann:052542, Friedrich Ludwig Finck:0526329}

\hypersetup{
  pdftitle={\titleline},
  pdfauthor={\authorsline},
  pdfsubject={\titleline - \subtitleline},
  pdfcreator={LuaLaTeX}
}

\begin{document}

% ------------------
% Titelseite
% ------------------
\begin{titlepage}
  \centering
  {\Large Technische Universität Berlin\par}
  \vspace{2cm}
  {\Huge\bfseries \titleline\par}
  \vspace{1.5cm}
  {\Large \subtitleline\par}
  \vfill
  {\large Autoren: \authorsline\par}
  \vspace{2cm}
  {\large \today\par}
  \thispagestyle{empty}
\end{titlepage}

% ------------------
% Inhaltsverzeichnis
% ------------------
\cleardoublepage
\pagenumbering{roman}
\setcounter{page}{1}

\tableofcontents
\clearpage

% ------------------
% Hauptteil
% ------------------
\cleardoublepage
\pagenumbering{arabic}
\setcounter{page}{1}

\chapter{Aufgabe} \label{ch:aufgabe}

\textbf{Geben Sie diese Hausaufgabe in schriftlicher Form in ISIS ab.}

Zeigen Sie, dass
\[ |\tan(x) + \tan(y)| \geq |x + y| \]
für alle $x, y \in \left(-\frac{\pi}{2}, \frac{\pi}{2}\right)$ gilt.

\vspace{1cm}

\begin{figure}[h!]
    \centering
    % Stellen Sie sicher, dass 'ha_09.png' im selben Verzeichnis liegt wie die .tex Datei
    \includegraphics[width=0.95\textwidth]{ha_09.png}
    \caption*{Original Aufgabenstellung}
\end{figure}

% ------------------
% Lösungen
% ------------------

\chapter{Lösung zur Aufgabe} \label{ch:loesung}

Um die Behauptung zu beweisen, nutzen wir den \textbf{Mittelwertsatz der Differentialrechnung} (MWS).

Sei $f: \left(-\frac{\pi}{2}, \frac{\pi}{2}\right) \to \mathbb{R}$ definiert durch $f(t) = \tan(t)$. Die Funktion $f$ ist auf ihrem gesamten Definitionsbereich differenzierbar mit der Ableitung:
\[ f'(t) = 1 + \tan^2(t) = \frac{1}{\cos^2(t)} \]

\textbf{Fallunterscheidung:}

\begin{enumerate}
    \item \textbf{Fall $x = -y$:} \\
    In diesem Fall gilt $x + y = 0$. Es folgt:
    \[ |\tan(x) + \tan(-x)| = |\tan(x) - \tan(x)| = 0 \geq 0 = |x - x| \]
    Die Ungleichung ist also trivialerweise erfüllt.

    \item \textbf{Fall $x \neq -y$:} \\
    Wir betrachten das Intervall zwischen $x$ und $-y$. Da $x, y \in \left(-\frac{\pi}{2}, \frac{\pi}{2}\right)$, liegt auch $-y$ in diesem Intervall. Ohne Beschränkung der Allgemeinheit nehmen wir an, dass $-y < x$. Da $f(t) = \tan(t)$ auf dem Intervall $[-y, x]$ stetig und auf $(-y, x)$ differenzierbar ist, existiert nach dem Mittelwertsatz ein $\xi \in (-y, x)$ mit:
    \[ \frac{f(x) - f(-y)}{x - (-y)} = f'(\xi) \]
    Einsetzen der Funktion und ihrer Ableitung ergibt:
    \[ \frac{\tan(x) - \tan(-y)}{x + y} = 1 + \tan^2(\xi) \]
    Unter Verwendung der Identität $\tan(-y) = -\tan(y)$ erhalten wir:
    \[ \frac{\tan(x) + \tan(y)}{x + y} = 1 + \tan^2(\xi) \]
    Durch Anwendung des Betrags folgt:
    \[ \left| \frac{\tan(x) + \tan(y)}{x + y} \right| = |1 + \tan^2(\xi)| \]
    Da $\tan^2(\xi) \geq 0$ für alle $\xi \in \mathbb{R}$ gilt, folgt $1 + \tan^2(\xi) \geq 1$. Somit:
    \[ \frac{|\tan(x) + \tan(y)|}{|x + y|} \geq 1 \]
    Multiplikation mit $|x + y|$ (da $x \neq -y$ ist $|x+y| > 0$) liefert die gewünschte Ungleichung:
    \[ |\tan(x) + \tan(y)| \geq |x + y| \]
\end{enumerate}

Damit ist die Behauptung für alle $x, y \in \left(-\frac{\pi}{2}, \frac{\pi}{2}\right)$ bewiesen.
\end{document}

% Local Variables:
% TeX-engine: luatex
% End:
