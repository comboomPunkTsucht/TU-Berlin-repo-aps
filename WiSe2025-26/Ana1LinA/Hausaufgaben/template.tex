% !TEX program = lualatex
\documentclass[12pt,a4paper]{report}

% --- Pakete ---
\usepackage{fontspec}        % Für lualatex/xelatex
\usepackage{microtype}       % Schönerer Satz
\usepackage{graphicx}        % Für Bilder
\usepackage{amsmath,amssymb} % Für mathematische Symbole und Umgebungen
\usepackage[ngerman]{babel}  % Deutsche Silbentrennung
\usepackage[hidelinks]{hyperref} % PDF-Links ohne Rahmen
\usepackage{bookmark}        % Besseres Inhaltsverzeichnis
\usepackage{fancyhdr}        % Für Kopf- und Fußzeile
\usepackage{nicefrac}        % Für schöne Brüche

% --- Schriftart (anpassen, falls gewünscht) ---
% Versuche zuerst eine Systemschrift, sonst Fallback
\IfFontExistsTF{CaskaydiaCove Nerd Font Propo}
  {\setmainfont{CaskaydiaCove Nerd Font Propo}}
  {\setmainfont{Latin Modern Roman}} % Fallback

% --- Kopf- und Fußzeile (Seitenzahl zentriert im Fuß) ---
\pagestyle{fancy}
\fancyhf{}
\fancyfoot[C]{\thepage}
\renewcommand{\headrulewidth}{0pt}
\renewcommand{\footrulewidth}{0pt}

% --- Metadaten und Konstanten (WICHTIG: Hier anpassen!) ---
\newcommand{\titleline}{Ana1LinA  - Hausaufgabe X}
\newcommand{\subtitleline}{Felix 09}
\newcommand{\authorsline}{{Fabian Aps:525528, Joris Victor Vorderwülbecke:0528715, Emil Arthur Joseph Hartmann:052542, Friedrich Ludwig Finck:0526329}}

% Eigener Befehl für die Funktion/das Polynom
% HIER FUNKTION ODER POLYNOM EINFÜGEN!
\newcommand{\mymathobject}{f(x) = \text{Funktion oder } p(x) = \text{Polynom}}

% Metadaten für PDF
\hypersetup{
  pdftitle={\titleline},
  pdfauthor={\authorsline},
  pdfsubject={\titleline - \subtitleline},
  pdfcreator={LuaLaTeX}
}

% --- Nummerierungsstil für enumerate-Listen ---
% (a), (b), ...
\renewcommand{\labelenumi}{\alph{enumi})}

\begin{document}

% ------------------
% Titelseite
% ------------------
\begin{titlepage}
  \centering
  {\Large Technische Universität Berlin\par}
  \vspace{2cm}
  {\Huge\bfseries \titleline\par}
  \vspace{1.5cm}
  {\Large \subtitleline\par}
  \vfill
  {\large Autoren: \authorsline\par}
  \vspace{2cm}
  {\large \today\par}
  \thispagestyle{empty} % Titelseite ohne Seitenzahl
\end{titlepage}

% ------------------
% Inhaltsverzeichnis (römische Seitenzahlen)
% ------------------
\cleardoublepage
\pagenumbering{roman}
\setcounter{page}{1}

\tableofcontents
\clearpage

% ------------------
% Hauptteil: arabische Seitenzahlen ab 1
% ------------------
\cleardoublepage
\pagenumbering{arabic}
\setcounter{page}{1}

\chapter{Aufgabenstellung}\label{ch:aufgaben}

Die gegebene Funktion/das Polynom lautet:
$$ \mymathobject $$ \label{func:main}

\begin{enumerate} % Nummerierung im Stil (a), (b), ...
    \item Hier kommt der Text von Aufgabe a).
    \item Hier kommt der Text von Aufgabe b).
    \item Hier kommt der Text von Aufgabe c).
    \item Hier kommt der Text von Aufgabe d).
\end{enumerate}

% Optional: Füge hier ein Bild der Aufgabenstellung ein
% \begin{figure}[h!]
%     \centerline{\includegraphics[width=1.0\textwidth,]{bildname.png}}
% \end{figure}

% ------------------
% Lösungen
% ------------------

\cleardoublepage % Beginnt Lösungs-Teil auf einer neuen Seite

\chapter{Lösung zu Aufgabe a)} \label{ch:a}
\textbf{Gegeben:} $ \mymathobject $
\par\bigskip
\textbf{Aufgabe:} \textit{Aufgabenstellung von a) in Kursiv.}
\par\bigskip

\section{Bearbeitung} \label{sec:aL}
\par
% HIER DEINE LÖSUNG EINFÜGEN. Verwende $$ für display-Gleichungen und \begin{align*} ... \end{align*} für Herleitungen.
Die Lösung zu Aufgabe a) ...

\clearpage
\chapter{Lösung zu Aufgabe b)} \label{ch:b}

\textbf{Gegeben:} $ \mymathobject $
\par\bigskip
\textbf{Aufgabe:} \textit{Aufgabenstellung von b) in Kursiv.}
\par\bigskip

\section{Bearbeitung} \label{sec:bL}
\par
% HIER DEINE LÖSUNG EINFÜGEN
Die Lösung zu Aufgabe b) ...

\clearpage
% ... Füge weitere Kapitel/Lösungen hinzu ...

\end{document}
