\documentclass[a4paper,10pt]{article}
\usepackage[utf8]{inputenc}
\usepackage[german]{babel}
\usepackage{amsmath, amssymb, amsfonts}
\usepackage{geometry}
\usepackage{multicol}
\usepackage{enumitem}
\usepackage{xcolor}
\usepackage{titlesec}

% ---------------------------------------------------------
% LAYOUT: Ränder minimiert für maximale Platzausnutzung
% ---------------------------------------------------------
\geometry{top=0.5cm, bottom=0.5cm, left=0.5cm, right=0.5cm}
\pagestyle{empty}

% Kompakte Listen
\setlist{nosep, leftmargin=*}

% Überschriften
\titlespacing*{\section}{0pt}{2pt}{1pt}
\titlespacing*{\subsection}{0pt}{2pt}{1pt}
\titleformat{\section}{\large\bfseries}{\thesection}{1em}{}
\titleformat{\subsection}{\normalsize\bfseries}{\thesubsection}{1em}{}

% Highlight-Box
\newcommand{\boxx}[1]{\fcolorbox{black}{gray!15}{\parbox{\linewidth}{#1}}}

\begin{document}

% ====================================================================
% SEITE 1: ANALYSIS (VL 1-9, 17-31)
% ====================================================================

\begin{center}
    \textbf{\Large Analysis I (VL 1--9, 17--31)}
\end{center}

\begin{multicols*}{2}

% ---------------------------------------------------------
\section*{1. Grundlagen \& Induktion (VL 1, 4)}
\textbf{Logik:} $A \Rightarrow B \iff \neg B \Rightarrow \neg A$ (Kontraposition) [42]. \\
\textbf{Binomischer Lehrsatz:} $(a+b)^n = \sum_{k=0}^n \binom{n}{k} a^{n-k}b^k$ [95].

\subsection*{Vollständige Induktion [85]}
Beweis einer Aussage $A(n)$ für alle $n \ge n_0$.
\begin{enumerate}
    \item \textbf{IA (Anfang):} Zeige $A(n_0)$ ist wahr.
    \item \textbf{IS (Schritt):} $n \to n+1$.
    \begin{itemize}
        \item \textit{IV (Voraussetzung):} Gelte $A(n)$.
        \item \textit{Schluss:} Zeige $A(n+1)$ unter Nutzung der IV.
    \end{itemize}
\end{enumerate}
\textbf{Wichtige Summen:} \\
$\sum_{k=1}^n k = \frac{n(n+1)}{2}$ (Gauß) [88] \quad |
\quad $\sum_{k=0}^n q^k = \frac{1-q^{n+1}}{1-q}$ (Geometr.) [66]

% ---------------------------------------------------------
\section*{2. Komplexe Zahlen $\mathbb{C}$ (VL 3, 7)}
$z = x + iy = r e^{i\phi} = r(\cos \phi + i \sin \phi)$. \\
$r = |z| = \sqrt{x^2+y^2}$, $\phi = \arg(z)$.
\begin{itemize}
    \item \textbf{Konjugation:} $\bar{z} = x - iy$. Es gilt $z\bar{z} = |z|^2$.
    \item \textbf{Euler-Formel:} $e^{i\phi} = \cos \phi + i \sin \phi$.
    \item \textbf{Multiplikation:} $z_1 z_2 = r_1 r_2 e^{i(\phi_1 + \phi_2)}$. (Winkel addieren!)
    \item \textbf{Potenzen (Moivre):} $z^n = r^n e^{in\phi}$.
    \item \textbf{Wurzeln:} $z^n = w = s e^{i\alpha}$ hat $n$ Lösungen ($k=0,\dots,n-1$):
    \[ z_k = \sqrt[n]{s} \cdot e^{i \frac{\alpha + 2\pi k}{n}} \quad \text{(Regelmäßiges $n$-Eck)} \]
\end{itemize}
\textbf{Quadrat. Gleichung:} $az^2+bz+c=0$.
Lsg: $z_{1,2} = \frac{-b \pm \sqrt{b^2-4ac}}{2a}$. (Wurzel aus komplexer Zahl beachten! Ggf. in Polar umwandeln).

% ---------------------------------------------------------
\section*{3. Folgen \& Stetigkeit (VL 17-20)}
\textbf{Grenzwerte:} $\lim \frac{1}{n}=0$, $\lim \sqrt[n]{n}=1$, $\lim (1+\frac{x}{n})^n = e^x$. \\
\textbf{Sandwich-Satz:} $a_n \le b_n \le c_n$ und $a_n, c_n \to a \Rightarrow b_n \to a$. \\
\textbf{Stetigkeit:} $f$ stetig in $x_0 \iff \lim_{x \to x_0} f(x) = f(x_0)$.
\begin{itemize}
    \item \textbf{Zwischenwertsatz:} $f$ stetig auf $[a,b]$, $f(a)<c<f(b) \Rightarrow \exists \xi: f(\xi)=c$. (Nullstellensuche).
    \item \textbf{Min/Max:} Stetige Fkt. auf kompaktem Intervall $[a,b]$ nimmt Min und Max an.
\end{itemize}

% ---------------------------------------------------------
\section*{4. Differentialrechnung (VL 21-25)}
\textbf{Ableitungen:} $(\sin x)' = \cos x$, $(\cos x)' = -\sin x$, $(\ln x)' = 1/x$, $(\arctan x)' = \frac{1}{1+x^2}$, $(\arcsin x)' = \frac{1}{\sqrt{1-x^2}}$. \\
\textbf{Regeln:} Produkt $(uv)'=u'v+uv'$, Quotient $(u/v)'=\frac{u'v-uv'}{v^2}$, Kette $(f(g(x)))' = f'(g(x))g'(x)$. \\
\textbf{Umkehrfunktion:} $(f^{-1})'(y) = \frac{1}{f'(f^{-1}(y))}$.
\subsection*{Mittelwertsatz \& l'Hospital}
\textbf{MWS:} Ist $f$ diffbar auf $]a,b[$, stetig auf $[a,b]$, dann $\exists \xi \in ]a,b[$:
\[ f'(\xi) = \frac{f(b)-f(a)}{b-a} \quad \text{(Steigung Tangente = Steigung Sekante)} \]
\textbf{L'Hospital:} Bei "$\frac{0}{0}$" oder "$\frac{\infty}{\infty}$": $\lim \frac{f(x)}{g(x)} = \lim \frac{f'(x)}{g'(x)}$.
\subsection*{Taylor \& Extrema (Wichtig!)}
\textbf{Extrema:} Notwendig $f'(x_0)=0$. Hinreichend:
\begin{itemize}
    \item $f''(x_0) > 0 \Rightarrow$ Min, $f''(x_0) < 0 \Rightarrow$ Max.
    \item Falls $f''(x_0)=0$: Erstes $f^{(n)}(x_0) \neq 0$ suchen. $n$ gerade? Extremum. $n$ ungerade? Sattelpunkt.
\end{itemize}
\boxx{
\textbf{Taylor-Formel mit Lagrange-Restglied (VL 24):}
\[ f(x) = \sum_{k=0}^n \frac{f^{(k)}(x_0)}{k!} (x-x_0)^k + R_n(x) \]
\[ R_n(x) = \frac{f^{(n+1)}(\xi)}{(n+1)!}(x-x_0)^{n+1}, \quad \xi \text{ zw. } x, x_0 \]
\textit{Reihen:} $e^x = \sum \frac{x^k}{k!}$, $\sin x = \sum (-1)^k \frac{x^{2k+1}}{(2k+1)!}$.
}

% ---------------------------------------------------------
\section*{5. Integralrechnung (VL 28-31)}
\textbf{Hauptsatz:} $\int_a^b f(x)dx = F(b) - F(a)$ mit $F' = f$.

\subsection*{Integrationstechniken}
\textbf{1. Partielle Integration:} $\int u' v = uv - \int u v'$.
(Tipp: Wähle $v$ so, dass es beim Ableiten einfacher wird, z.B. $x \cdot e^x$).

\textbf{2. Substitution:} $x = g(t) \Rightarrow dx = g'(t)dt$.
\[ \int_{g(a)}^{g(b)} f(x) dx = \int_a^b f(g(t)) g'(t) dt \]

\subsection*{Partialbruchzerlegung (PBZ) - VL 9/31}
Ziel: Integriere rationale Funktion $R(x) = \frac{P(x)}{Q(x)}$.
\begin{enumerate}
    \item \textbf{Polynomdivision:} Falls Grad $P \ge$ Grad $Q$ $\to$ Rest $\frac{R(x)}{Q(x)}$.
    \item \textbf{Nullstellen Nenner:} $Q(x)$ faktorisieren.
    \item \textbf{Ansatz:}
    \begin{itemize}
        \item \textit{Einfach} $(x-x_0)$: $\frac{A}{x-x_0}$. ($\int \to A \ln|x-x_0|$)
        \item \textit{Mehrfach} $(x-x_0)^k$: $\frac{A_1}{x-x_0} + \dots + \frac{A_k}{(x-x_0)^k}$. \\
        ($\int \to \ln$ und Potenzen $\frac{1}{(1-k)(x-x_0)^{k-1}}$).
        \item \textit{Komplex} $(x^2+px+q)$: $\frac{Bx+C}{x^2+px+q}$. \\
        $\to$ Zerlegen in $\frac{2x+p}{x^2+px+q}$ ($\int \to \ln(\text{Nenner})$) und Rest ($\to \arctan$).
    \end{itemize}
    \item \textbf{Koeffizienten:} Zuhaltemethode (einfache Pole) oder LGS.
\end{enumerate}

\subsection*{Uneigentliche Integrale (VL 31)}
Grenzen $\pm \infty$ oder Polstellen. Als Grenzwert berechnen!
$\int_a^\infty f(x) dx = \lim_{b \to \infty} \int_a^b f(x) dx$.
Wichtig: $\int_1^\infty \frac{1}{x^\alpha} dx$ konv. für $\alpha > 1$. $\int_0^1 \frac{1}{x^\alpha} dx$ konv. für $\alpha < 1$.
\end{multicols*}

\newpage

% ====================================================================
% SEITE 2: LINEARE ALGEBRA (VL 10-16, 32, 33)
% ====================================================================

\begin{center}
    \textbf{\Large Lineare Algebra (VL 10--16, 32--33)}
\end{center}

\begin{multicols*}{2}

% ---------------------------------------------------------
\section*{6. Vektorräume (VL 10-11)}
\textbf{Unterraum:} $U \subset V$ ist UR, wenn $0 \in U$ und abgeschlossen bzgl. $+$ und $\cdot$ (Skalar). \\
\textbf{Linear Unabhängig:} $\sum \lambda_i v_i = 0 \Rightarrow \forall \lambda_i = 0$. \\
\textbf{Basis:} Linear unabhängiges Erzeugendensystem. \\
\textbf{Dimension:} Anzahl der Basisvektoren. \\
\textbf{Koordinaten:} $v = \sum \lambda_i b_i \Rightarrow$ Koord-Vektor bzgl Basis $B$ ist $(\lambda_1, \dots, \lambda_n)^T$.

% ---------------------------------------------------------
\section*{7. Matrizen \& LGS (VL 12-14)}
\textbf{Rang:} Anzahl linear unabhängiger Zeilen/Spalten (oder Pivot-Elemente in der Zeilenstufenform ZSF). \\
\textbf{Lösbarkeit von $Ax=b$:}
\begin{itemize}
    \item \textit{Keine Lsg:} $\text{Rang}(A) < \text{Rang}(A|b)$.
    \item \textit{Genau eine Lsg:} $\text{Rang}(A) = \text{Rang}(A|b) = n$ (Variablenanzahl).
    \item \textit{Unendl. Lsg:} $\text{Rang}(A) = \text{Rang}(A|b) < n$. \\ (Freie Parameter = $n - \text{Rang}(A)$).
\end{itemize}
\textbf{Inverse:} $A \in K^{n,n}$ invertierbar $\iff \det(A) \neq 0 \iff \text{Rang}(A)=n$. \\
Berechnung: $(A | I_n) \xrightarrow{\text{Gauss}} (I_n | A^{-1})$. \\
Regel: $(AB)^{-1} = B^{-1}A^{-1}$, $(A^T)^{-1} = (A^{-1})^T$.

% ---------------------------------------------------------
\section*{8. Lineare Abbildungen (VL 15-16)}
$f: V \to W$ linear, $f(\lambda u + v) = \lambda f(u) + f(v)$. \\
\textbf{Kern:} $\text{Kern}(f) = \{v \in V \mid f(v)=0\}$. (Lösungsraum von $Ax=0$). \\
\textbf{Bild:} $\text{Bild}(f) = \{f(v) \mid v \in V\}$. (Spaltenraum von $A$). \\
Injektiv $\iff \text{Kern}=\{0\}$. Surjektiv $\iff \text{Bild}=W$.

\boxx{
\textbf{Dimensionsformel:}
\[ \dim(V) = \dim(\text{Kern}(f)) + \dim(\text{Bild}(f)) \]
Für Matrix $A \in K^{m,n}$: $n = \dim(\text{Kern}) + \text{Rang}(A)$.
}

\subsection*{Matrixdarstellung \& Basiswechsel}
Sei $B=\{b_1,\dots,b_n\}$ Basis von $V$, $C$ Basis von $W$. \\
Matrix $A = M_C^B(f)$: Spalte $j$ enthält Koord. von $f(b_j)$ bzgl. $C$.
\[ f(v)_C = A \cdot v_B \]
\textbf{Basiswechsel:} Seien $B, B'$ Basen von $V$. Transformationsmatrix $T = M_B^{B'}(id)$ (Spalten sind $b'_j$ dargestellt in $B$). \\
Koord-Transf: $v_B = T \cdot v_{B'}$. \\
\textbf{Matrix-Transformation:} Sei $A$ Matrix bzgl. $B$ und $A'$ Matrix bzgl. $B'$ (bei Endomorphismus $V \to V$):
\[ A' = T^{-1} A T \quad (\text{ähnliche Matrizen}) \]

% ---------------------------------------------------------
\section*{9. Determinanten (VL 32)}
\begin{itemize}
    \item $2\times2$: $\det \begin{pmatrix} a & b \\ c & d \end{pmatrix} = ad-bc$.
    \item $3\times3$: Sarrus (Jägerzaun / Diagonalen).
    \item \textbf{Laplace-Entwicklung:} Nach Zeile $i$ oder Spalte $j$:
    $\det(A) = \sum_{k=1}^n (-1)^{i+k} a_{ik} \det(A_{ik})$. \\
    (Tipp: Zeile/Spalte mit vielen Nullen wählen).
    \item \textbf{Dreiecksmatrix:} Produkt der Diagonalelemente.
    \item \textbf{Eigenschaften:}
    \begin{itemize}
        \item $\det(A) = \det(A^T)$.
        \item $\det(AB) = \det(A)\det(B)$.
        \item $\det(A^{-1}) = 1/\det(A)$.
        \item Zeilentausch: Vorzeichenwechsel.
        \item Zeile mal $\lambda$: Determinante mal $\lambda$. ($\det(\lambda A) = \lambda^n \det(A)$).
    \end{itemize}
\end{itemize}

% ---------------------------------------------------------
\section*{10. Eigenwerte (EW) und Eigenvektoren (EV) (VL 33)}
Gleichung: $A v = \lambda v$ mit $v \neq 0$.
\subsection*{Algorithmus zur Berechnung}
\begin{enumerate}
    \item \textbf{Charakteristisches Polynom:}
    \[ p_A(\lambda) = \det(A - \lambda I_n) \]
    \item \textbf{Eigenwerte bestimmen:} Nullstellen von $p_A(\lambda)$.
    \item \textbf{Algebraische Vielfachheit $a(\lambda)$:}
    Vielfachheit der Nullstelle im Polynom $(X-\lambda)^k$.
    \item \textbf{Eigenraum / Eigenvektoren:}
    Löse das homogene LGS für jedes $\lambda$:
    \[ (A - \lambda I_n)v = 0 \]
    Der Lösungsraum ist der Eigenraum $V_\lambda$. Basis berechnen!
    \item \textbf{Geometrische Vielfachheit $g(\lambda)$:}
    \[ g(\lambda) = \dim(V_\lambda) = n - \text{Rang}(A - \lambda I_n) \]
\end{enumerate}

\boxx{
\textbf{Eigenschaften:}
\begin{itemize}
    \item $1 \le g(\lambda) \le a(\lambda)$.
    \item $\sum a(\lambda_i) = n$ (im Komplexen immer).
    \item Spur$(A) = \sum a_{ii} = \sum \lambda_i$ (Summe inkl. Vielfachheit).
    \item $\det(A) = \prod \lambda_i$ (Produkt inkl. Vielfachheit).
\end{itemize}
}
\textit{Beispiel:} $A = \begin{pmatrix} 1 & 1 \\ 0 & 1 \end{pmatrix}$. $p_A(\lambda) = (1-\lambda)^2$. EW $\lambda=1$ mit $a(1)=2$.
$A-1I = \begin{pmatrix} 0 & 1 \\ 0 & 0 \end{pmatrix}$. Rang=1. $g(1) = 2 - 1 = 1$. Hier gilt $g(1) < a(1)$.

\end{multicols*}
\end{document}
